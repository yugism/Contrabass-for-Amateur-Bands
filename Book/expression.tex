\section{表現力のトレーニング}
1年目用の教材ではそれぞれの音がどの位置にあるのかを学びました。2年目は単に音を出すのではなく、美しい音で歌うためのテクニックを習得します。

\subsection{ヴィブラート(伊: vibrato)}
音の低い方へ向かって指・手首を揺らす。
pizz.の音には必ずヴィブラートをかけて響きを増幅する。

\subsubsection*{シューベルト: 交響曲第8番ロ短調「未完成」 第1楽章 冒頭}

\begin{music}
\setclef1{\bass}
\generalsignature{2}    
\generalmeter{\meterfrac34}
\parindent 0pt
\nostartrule
\startpiece\bigaccid
\qspace
\Notes\zchar{12}{\bf Allegro moderato}\zchar{-5}{\pp}\islurd{0}{'B}\hup{B}\enotes
\leftrepeat
\NOtes\hu{'C}\enotes
\Notes\tslur{0}{'D}\ql{D}\enotes
\bar
\NOtes\islurd{0}{'A}\hup{B}\enotes
\bar
\Notes\qu{'A}\qu{!F}\tslur{0}{F}\qu{G}\enotes
\bar
\NOtes\isluru{0}{'G}\hl{D}\enotes
\Notes\qu{'C}\enotes
\bar
\NOtes\tslur{0}{a}\islurd{1}{F}\hup{F}\enotes
\bar
\NOtes\tslur{1}{F}\islurd{0}{F}\hup{F}\enotes
\bar
\NOtes\tslur{0}{F}\hup{F}\enotes
\setdoublebar
\endpiece
\end{music}


\subsection{レガート(伊: legato)}
音の変わり目を切らずにつなぐこと。移弦のときには移る先の音を押さえておく。音を出す前にヴィブラート。ブラ2の2楽章82、を移弦の練習に

\subsubsection*{シベリウス: 交響曲第2番 第4楽章より}

\documentclass{jarticle}
\usepackage{musixdoc}
\startmuflex\makeindex

\begin{document}

\subsubsection*{�١��ȡ�������: �������5�� ��ûĴ ��3�ھϤ��}

\begin{music}
\setclef1{\bass}
\generalsignature{-3}    
\generalmeter{\meterfrac34}
\parindent 0pt
\nostartrule
\startpiece\bigaccid
\qspace
\NOtes\zchar{18}{\bf Allegro (\metron{\hu}{96})}\zchar{-5}{\pp}\isluru{0}{'G}\qu{!G}\enotes
\bar
\Notes\qu{'C}\ql{EG}\enotes
\bar
\NOtes\hl{c}\enotes
\Notes\ql{e}\enotes
\bar
\NOtes\hl{d}\midslur{8}\enotes
\Notes\tslur{0}{'G}\ql{^F}\enotes
\bar
\NOTes\isluru{0}{'G}\hlp{G}\enotes
\bar
\NOTes\tslur{0}{'G}\hlp{G}\enotes
\bar
\NOTes\islurd{0}{'C}\hup{C}\enotes
\bar
\NOTes\tslur{0}{'C}\zchar{12}{\it poco rit.}\hup{C}\enotes
\alaligne
\NOtes\Fermataup{7}\roffset{5.5}{\zchar{16}{\it a tempo}}\hu{G}\enotes
\Notes\isluru{0}{a}\zchar{-5}{\pp}\qu{G}\enotes
\bar
\Notes\qu{'C}\ql{EG}\enotes
\bar
\NOtes\hl{c}\enotes
\Notes\ql{e}\midslur{8}\enotes
\bar
\NOtes\hl{d}\enotes
\Notes\ql{'^F}\enotes
\bar
\NOtes\hl{'G}\enotes
\Notes\tslur{0}{'E}\qu{^C}\enotes
\bar
\NOTes\isluru{0}{'D}\zchar{-5}{\ppfftwenty sfp}\hlp{D}\enotes
\alaligne
\NOtes\tslur{0}{'D}\hl{D}\enotes
\Notes\islurd{0}{'C}\qu{=C}\enotes
\bar
\NOtes\hu{'=B}\enotes
\Notes\qu{G}\midslur{-5}\enotes
\bar
\NOtes\hu{'C}\enotes
\Notes\tslur{0}{'B}\qu{=B}\enotes
\bar
\NOtes\islurd{0}{'B}\zchar{12}{\it poco rit.}\lpz{C}\hu{C}\enotes
\Notes\tslur{0}{E}\lpz{F}\qu{^F}\enotes
\bar
\NOTes\Fermataup{7}\hup{G}\enotes
\setdoublebar
\endpiece
\end{music}

\endmuflex
\end{document}

\documentclass{jarticle}
\usepackage{musixdoc}
\startmuflex\makeindex

\newcommand{\flag}[2]{\zchar{#2}{$\circ$}\zchar{#1}{\tiny +}}
\newcommand{\press}[2]{\zchar{#2}{$\circ$}\zchar{#1}{\small \|}}

\begin{document}

\subsubsection*{�١��ȡ�������: �������9�֥�ûĴ�ֹ羧�դ��� ��4�ھϤ��}

\begin{music}
\setclef1{\bass}
\generalsignature{2}    
\generalmeter{\meterfrac64}
\nostartrule
\parindent 0pt
\startbarno=719
\def\writebarno{\tenrm\the\barno\barnoadd}
\def\raisebarno{2\internote}
\def\shiftbarno{0.1\Interligne}
\systemnumbers
\startpiece\bigaccid
\qspace
\notes\zchar{14}{\bf Allegro energico, sempre ben marcato (\metron{\hup}{84})}\enotes
\NOtes\hpausep\hlp{d}\enotes
\bar
\NOtes\hlp{d}\hlp{d}\enotes
\bar
\NOtes\hlp{c}\hlp{a}\enotes
\bar
\NOtes\hlp{b}\hlp{b}\enotes
\bar
\NOtes\hlp{a}\hlp{'F}\enotes
\alaligne
\NOtes\hlp{'G}\hlp{G}\enotes
\bar
\NOtes\hlp{'F}\hlp{!d}\enotes
\bar
\NOtes\hlp{e}\hlp{e}\enotes
\bar
\NOtes\whp{e}\enotes
\bar
\NOtes\hlp{e}\hlp{e}\enotes
\bar
\NOtes\whp{d}\enotes
\setdoublebar
\endpiece
\end{music}

\endmuflex
\end{document}




\subsection{アクセント}

\documentclass{jarticle}
\usepackage{musixdoc}
\startmuflex\makeindex

\begin{document}

\subsubsection*{�ޡ��顼: �������2�� ��ûĴ ������� ��1�ھϤ��}

\begin{music}
\nostartrule
\setclef1{\bass}
\generalsignature{-4}    
\generalmeter{\meterfrac44}
\parindent 0pt
\startbarno=1
\def\writebarno{\tenrm\the\barno\barnoadd}
\def\raisebarno{2\internote}
\def\shiftbarno{0.1\Interligne}
\systemnumbers
\startpiece\bigaccid
\Notes\hp\qp\ibu{0}{'B}{2}\qb{0}{=B}\tbu{0}\qb{0}{C}\enotes
\bar
\Notes\hl{'D}\ibu{0}{D}{-1}\qb{0}{DC=B}\tbu{0}\qb{0}{C}\enotes
\bar
\Notes\qu{'A}\qu{!F}\qp\ibu{0}{'B}{2}\qb{0}{=B}\tbu{0}\qb{0}{C}\enotes
\bar
\Notes\qlp{'D}\cl{C}\ibu{0}{B}{2}\qb{0}{=BCF}\tbu{0}\qb{0}{!a}\enotes
\bar
\Notes\hl{c}\ibl{0}{c}{0}\qb{0}{cab}\tbl{0}\qb{0}{c}\enotes
\bar
\Notes\ql{d}\qp\hp\enotes
\bar
\Notes\hp\qp\ql{a}\enotes
\bar
\Notes\hl{d}\ibl{0}{d}{0}\qb{0}{dcb}\tbl{0}\qb{0}{c}\enotes
\bar
\Notes\ql{d'=G}\qp\ql{G}\enotes
\bar
\Notes\hl{d}\ibl{0}{d}{0}\qb{0}{dcb}\tbl{0}\qb{0}{c}\enotes
\bar
\Notes\ql{^c=d}\qp\ibl{0}{d}{0}\qb{0}{c}\tbl{0}\qb{0}{d}\enotes
\bar
\Notes\ql{^d=e}\qp\ibl{0}{a}{0}\qb{0}{=a}\tbl{0}\qb{0}{=d}\enotes
\bar
\Notes\ql{^d=e}\qp\ibl{0}{a}{0}\qb{0}{=a}\tbl{0}\qb{0}{=e}\enotes
\bar
\Notes\hl{f}\ibl{0}{f}{0}\qb{0}{f=e=d}\tbl{0}\qb{0}{e}\enotes
\bar
\Notes\ql{fc}\qp\ibl{0}{e}{0}\qb{0}{=e}\tbl{0}\qb{0}{f}\enotes
\bar
\Notes\hl{_g}\ibl{0}{g}{0}\qb{0}{gf=e}\tbl{0}\qb{0}{f}\enotes
\bar
\Notes\ql{_g}\ql{c}\qp\ql{g}\enotes
\bar
\Notes\ql{c}\qp\ql{_gc}\enotes
\bar
\Notes\qp\ql{_gc}\qp\enotes
\bar
\Notes\hl{_g}\cl{c}\ds\qp\enotes
\bar
\Notes\hl{_g}\cl{c}\ds\qp\enotes
\bar
\Notes\wh{_g}\enotes
\bar
\Notes\hl{_g}\ql{f=e_e=d}\enotes
\bar
\Notes\cl{^c}\ds\hup{'C}\enotes
\setdoublebar
\endpiece
\end{music}

\endmuflex
\end{document}


\subsection{デタシェ}




\subsection{音量法と弓の使い方}
開放弦の練習がここで役立つ

弓先、弓元の使い分けと音量記号

駒からの距離

\documentclass{jarticle}
\usepackage{musixdoc}
\startmuflex\makeindex

\newcommand{\flag}[2]{\zchar{#2}{$\circ$}\zchar{#1}{\tiny +}}
\newcommand{\press}[2]{\zchar{#2}{$\circ$}\zchar{#1}{\small \|}}

\begin{document}

\subsubsection*{�������С�: �η�����Ƥμͼ�׽��� ��Ƭ}

\begin{music}
\setclef1{\bass}
\generalsignature{0}    
\generalmeter{\meterC}
\parindent 0pt
\nostartrule
\startpiece\bigaccid
\qspace\qspace
\NOtes\zchar{14}{\bf Adagio}\zchar{-6}{\pp}\zchar{8}{\upbow}\isluru{0}{'F}\wh{C}\zchar{-5}{\icresc}\enotes
\bar
\Notes\zchar{-5}{\tcresc}\zchar{-6}{\f}\zchar{15}{\downbow}\qlp{c}\enotes
\notes\cl{b}\enotes
\Notes\tslur{0}{d}\ql{d}\qp\enotes
\bar
\Notes\zchar{8}{\Large\bf \ 2}\PAuse\enotes
\bar
\NOtes\zchar{-6}{\pp}\isluru{0}{'D}\zchar{8}{\upbow}\wh{!G}\zchar{-5}{\icresc}\enotes
\bar
\Notes\zchar{-5}{\tcresc}\zchar{-6}{\f}\zchar{14}{\downbow}\qlp{'G}\enotes
\notes\cl{'F}\enotes
\Notes\tslur{0}{a}\ql{a}\qp\enotes
\setdoublebar
\endpiece
\end{music}

\endmuflex
\end{document}



\subsubsection*{ブラームス: 交響曲第1番 ハ短調 第1楽章より}


\subsubsection*{リスト: 交響詩「前奏曲」より}

各フレーズのトップノートを少し強め、vib.多めにする。

\begin{music}
\nostartrule
\setclef1{\bass}
\generalmeter{\meterC}
\parindent 0pt
%\startbarno=29
\def\writebarno{\tenrm\the\barno\barnoadd}
\def\raisebarno{2\internote}
\def\shiftbarno{0.1\Interligne}
\systemnumbers
\startpiece\bigaccid
\notes\zchar{18}{\bf Andante}\enotes
\Notes\hpause\zchar{11}{pizz.}\zchar{-5}{\p}\qu{'C}\qp\enotes
\bar
\Notes\hpause\qu{'C}\qp\enotes
\bar
\Notes\qp\zchar{11}{arco}\islurd{0}{'B}\qup{C}\enotes
\Notes\cu{'B}\isluru{1}{E}\ql{E}\enotes
\bar
\Notes\ibu{0}{'E}{-4}\tslur{1}{E}\midslur{-5}\qb{0}{E}\qb{0}{C}\qb{0}{A}\tbu{0}\qb{0}{!G}\enotes
\Notes\ibu{0}{'A}{3}\qb{0}{A}\tbu{0}\tslur{0}{B}\qb{0}{C}\enotes
\Notes\qp\enotes
\bar
\Notes\loffset{0.8}{\zchar{9}{\Large\bf \ 1}}\pause\enotes
\bar
\Notes\loffset{1.0}{\zchar{14}{\it poco riten.}}\roffset{0.35}{\zchar{9}{\Large\bf \ 3}}\PAuse\pause\enotes
\bar
\Notes\hpause\qp\Fermataup{7}\qp\enotes
\bar
\Notes\loffset{2.8}{\zchar{10}{\LARGE\bf A}}\hpause\zchar{11}{pizz.}\zchar{-5}{\p}\ql{'D}\qp\enotes
\bar
\Notes\hpause\qu{'D}\qp\enotes
\bar
\Notes\qp\zchar{11}{arco}\isluru{0}{'D}\qlp{D}\enotes
\Notes\cu{'^C}\isluru{1}{F}\ql{F}\enotes
\bar
\Notes\ibl{0}{'F}{-4}\tslur{1}{F}\midslur{7}\qb{0}{F}\qb{0}{D}\qb{0}{_B}\tbl{0}\qb{0}{A}\enotes
\Notes\ibl{0}{'B}{2}\qb{0}{B}\qb{0}{^C}\tbl{0}\tslur{0}{D}\qb{0}{D}\enotes
\Notes\ds\enotes
\bar
\Notes\loffset{0.8}{\zchar{9}{\Large\bf \ 1}}\pause\enotes
\bar
\Notes\loffset{1.0}{\zchar{14}{\it poco riten.}}\roffset{0.35}{\zchar{9}{\Large\bf \ 4}}\PAuse\pause\enotes
\bar
\Notes\hpause\qp\islurd{0}{F}\islurd{1}{G}\zchar{-5}{\p}\lst{'A}\qup{_A}\enotes
\bar
\Notes\ibu{0}{'A}{-2}\tslur{1}{A}\qb{0}{A}\tbu{0}\qb{0}{!G}\enotes
\Notes\islurd{1}{'D}\qu{D}\enotes
\Notes\ibu{0}{'D}{-3}\tslur{1}{D}\midslur{-4}\qb{0}{D}\qb{0}{B}\qb{0}{_A}\tbu{0}\qb{0}{!G}\enotes
\bar
\Notes\zchar{-6}{\icresc}\ibu{0}{'A}{4}\qb{0}{_A}\qb{0}{B}\qb{0}{D}\tbu{0}\tslur{0}{B}\zchar{-6}{\tcresc}\qb{0}{F}\enotes
\Notes\hpause\enotes
\bar
\Notes\hpause\qp\islurd{0}{G}\zchar{-5}{\p}\lst{'A}\qup{=A}\enotes
\bar
\Notes\ibu{0}{'A}{-1}\tslur{0}{A}\islurd{0}{A}\qb{0}{A}\tbu{0}\qb{0}{!G}\enotes
\Notes\isluru{1}{'E}\ql{_E}\enotes
\Notes\ibu{0}{'E}{-4}\tslur{1}{E}\qb{0}{E}\qb{0}{C}\qb{0}{A}\tbu{0}\midslur{-4}\qb{0}{!G}\enotes
\bar
\Notes\zchar{-5}{\icresc}\ibu{0}{'A}{5}\qb{0}{A}\qb{0}{C}\qb{0}{_E}\tbu{0}\tslur{0}{D}\zchar{-5}{\tcresc}\qb{0}{^F}\enotes
\Notes\hpause\enotes
\bar
\Notes\hpause\islurd{0}{'B}\qup{_B}\enotes
\notes\cu{G}\enotes
\bar
\Notes\ibu{0}{'E}{-4}\qb{0}{=E}\qb{0}{^C}\qb{0}{_B}\tbu{0}\midslur{-4}\qb{0}{!G}\enotes
\Notes\ibu{0}{'B}{3}\qb{0}{B}\qb{0}{C}\tbu{0}\tslur{0}{E}\qb{0}{E}\ds\enotes
\bar
\Notes\hpause\islurd{0}{'B}\zchar{-6}{\it poco a poco crescendo -   -   -   -   -   -   -   -   -   -   -   -}\qup{=B}\enotes
\notes\cu{G}\enotes
\bar
\Notes\ibu{0}{'E}{-4}\qb{0}{E}\qb{0}{C}\qb{0}{B}\tbu{0}\midslur{-4}\qb{0}{!G}\enotes
\Notes\ibu{0}{'B}{3}\qb{0}{B}\qb{0}{^D}\tbu{0}\tslur{0}{E}\qb{0}{E}\ds\enotes
\bar
\notes\zchar{-4}{\it pi\`{u} crescendo -   -   -   -   -   -   -   -   -   -   -   -   -   -   -}\enotes
\Notes\hp\isluru{0}{a}\ibu{0}{'B}{-3}\qb{0}{B}\tbbu{0}\tbu{0}\qb{0}{!G}\tslur{0}{a}\ql{'F}\enotes
\bar
\Notes\hp\isluru{0}{a}\ibu{0}{'B}{-3}\qb{0}{B}\tbbu{0}\tbu{0}\qb{0}{!G}\tslur{0}{'F}\isluru{0}{G}\usf{F}\ql{F}\enotes
\bar
\Notes\zchar{-6}{\it -   -   -   -   -   -   -   -   -   -   -   -   -   -   -   -   -   -   -   -   -   -   -   -   -   -   -   -   -   -   -   -   -   -   -   -   -   -   -   -   -   -   -   -   -   -   -   -   -   -   -   -   -   -}\tslur{0}{'G}\ibu{0}{F}{-3}\qb{0}{F}\isluru{0}{!e}\qb{0}{'D}\qb{0}{B}\tbu{0}\qb{0}{A}\enotes
\Notes\ibu{0}{'B}{3}\zchar{-2}{\icresc}\qb{0}{!G}\qb{0}{'A}\qb{0}{B}\tbu{0}\qb{0}{D}\enotes
\bar
\Notes\midslur{7}\ibl{0}{'F}{3}\qb{0}{F}\qb{0}{G}\qb{0}{!b}\tbl{0}\qb{0}{d}\enotes
\Notes\ibl{0}{d}{1}\qb{0}{f}\qb{0}{g}\qb{0}{'a}\tbl{0}\zchar{-2}{\tcresc}\qb{0}{!g}\enotes
\bar
\Notes\ibl{0}{e}{-3}\qb{0}{f}\qb{0}{d}\qb{0}{b}\tbl{0}\qb{0}{a}\enotes
\Notes\ibl{0}{'G}{-3}\qb{0}{G}\qb{0}{F}\qb{0}{D}\tbl{0}\qb{0}{B}\enotes
\bar
\notes\zchar{-6}{\it -   -   -   -   -   -}\enotes
\Notes\ibu{0}{'F}{0}\qb{0}{A}\tslur{0}{!f}\qb{0}{!G}\zchar{-5}{\f}\lpz{'F}\qb{0}{F}\tbu{0}\lpz{D}\qb{0}{D}\enotes
\Notes\ibu{0}{'A}{0}\lpz{B}\qb{0}{B}\lpz{!G}\qb{0}{G}\lpz{'A}\qb{0}{A}\tbu{0}\lpz{B}\qb{0}{B}\enotes
\generalmeter{\meterfrac{12}{8}\zchar{2}{\Huge (} \meterC\zchar{2}{\Huge )}}\Changecontext 
\notes\zchar{16}{\bf Andante maestoso}\zchar{-5}{\ff}\cu{'C}\ds\ds\enotes
\notes\zchar{3}{\Huge (}\enotes
\Notes\lsf{G}\isluru{0}{!c}\isluru{1}{c}\ql{c}\enotes
\notes\ibl{0}{c}{-1}\ibbl{0}{c}{-1}\tslur{1}{c}\qb{0}{c}\tbbl{0}\tbl{0}\tslur{0}{b}\qb{0}{b}\enotes
\notes\ibl{0}{e}{-4}\upz{e}\qb{0}{e}\upz{c}\qb{0}{c}\tbl{0}\upz{b}\qb{0}{b}\enotes
\notes\ibl{0}{'G}{0}\upz{!a}\qb{0}{a}\upz{'G}\qb{0}{G}\tbl{0}\upz{!a}\qb{0}{a}\enotes
\notes\zchar{3}{\Huge )}\enotes
\Endpiece
\end{music}


\clearpage
\section{運動神経のトレーニング}
\subsection{高速系}
速いパッセージ: シュマ4フィナーレ  チャイ4  ブル4  ベト3  メンデルスゾーン  ウェーバー系  

\section{2年目は表現力UP: はね弓とヴィブラート}
\subsection{はね弓(1) スピッカート (伊: spiccato)}

弓の重心を探す(写真)

重心だけで着地→離陸

弦に対して垂直に弓をぶつける

ブラ2


\subsubsection*{モーツァルト: 歌劇「魔笛」序曲より}

\begin{music}
\setclef1{\bass}
\generalsignature{-3}    
\generalmeter{\allabreve}
\parindent 0pt
\startbarno=33
\def\writebarno{\tenrm\the\barno\barnoadd}
\def\raisebarno{2\internote}
\def\shiftbarno{0.1\Interligne}
\systemnumbers
\nostartrule
\startpiece\bigaccid
\qspace
\notes\zchar{12}{\bf (Allegro)}\zchar{-5}{\p}\enotes
\Notes\ibu{0}{'B}{0}\lppz{B}\qb{0}{B}\lppz{B}\qb{0}{B}\lppz{B}\qb{0}{B}\tbu{0}\lppz{B}\qb{0}{B}\enotes
\Notes\ibu{0}{'B}{0}\lppz{B}\qb{0}{B}\tbu{0}\lppz{B}\qb{0}{B}\enotes
\notes\zchar{-5}{\f}\ibu{0}{'C}{-1}\ibbu{0}{C}{-1}\islurd{0}{C}\qb{0}{C}\qb{0}{B}\qb{0}{=A}\tbbu{0}\tbu{0}\tslur{0}{B}\qb{0}{B}\enotes
\bar
\NOtes\zchar{-5}{\p}\loffset{0.67}{\uppz{'F}}\loffset{0.25}{\uppz{F}}\roffset{0.25}{\uppz{F}}\roffset{0.67}{\uppz{F}}\loffset{0.5}{\ibl{0}{G}{9}}\roffset{0.5}{\tbl{0}}\hl{F}\enotes
\Notes\ibl{0}{'F}{0}\uppz{F}\qb{0}{F}\tbl{0}\uppz{F}\qb{0}{F}\enotes
\notes\zchar{-6}{\f}\ibl{0}{'E}{0}\ibbl{0}{E}{0}\isluru{0}{G}\qb{0}{G}\qb{0}{F}\qb{0}{=E}\tbbl{0}\tbl{0}\tslur{0}{F}\qb{0}{F}\enotes
\bar
\Notes\zchar{-6}{\p}\ibl{0}{'D}{1}\uppz{D}\qb{0}{D}\uppz{D}\qb{0}{D}\uppz{G}\qb{0}{G}\tbl{0}\uppz{G}\qb{0}{G}\enotes
\Notes\ibl{0}{'C}{1}\uppz{C}\qb{0}{C}\uppz{C}\qb{0}{C}\zchar{-7}{\ppfftwenty sf}\uppz{F}\qb{0}{F}\tbl{0}\uppz{F}\qb{0}{F}\enotes
\bar
\Notes\zchar{-6}{\p}\ibl{0}{'D}{1}\uppz{D}\qb{0}{D}\uppz{D}\qb{0}{D}\uppz{G}\qb{0}{G}\tbl{0}\uppz{G}\qb{0}{G}\enotes
\Notes\ibl{0}{'C}{1}\uppz{C}\qb{0}{C}\uppz{C}\qb{0}{C}\zchar{-7}{\ppfftwenty sf}\uppz{F}\qb{0}{F}\tbl{0}\uppz{F}\qb{0}{F}\enotes
\bar
\NOTes\zchar{-4}{\p}\wh{b}\enotes
\bar
\NOTes\wh{'B}\enotes
\bar
\Notes\zchar{-5}{\f}\ql{'E}\qp\enotes
\NOtes\zchar{-5}{\ppfftwenty sf}\isluru{0}{e}\hl{e}\enotes
\bar
\NOtes\tslur{0}{e}\ql{e}\enotes
\Notes\ibl{0}{d}{-2}\isluru{0}{e}\upz{d}\qb{0}{d}\tbl{0}\tslur{0}{d}\upz{c}\qb{0}{c}\enotes
\Notes\ibl{0}{b}{-2}\uppz{b}\qb{0}{b}\uppz{a}\qb{0}{a}\uppz{'G}\qb{0}{G}\tbl{0}\uppz{F}\qb{0}{F}\enotes
\bar
\Notes\ibl{0}{'E}{0}\uppz{E}\qb{0}{E}\uppz{E}\qb{0}{E}\uppz{E}\qb{0}{E}\tbl{0}\uppz{E}\qb{0}{E}\enotes
\Notes\ibl{0}{'E}{0}\uppz{E}\qb{0}{E}\tbl{0}\uppz{E}\qb{0}{E}\enotes
\notes\ibl{0}{'D}{0}\ibbl{0}{D}{0}\isluru{0}{F}\qb{0}{F}\qb{0}{E}\qb{0}{D}\tbbl{0}\tbl{0}\tslur{0}{E}\qb{0}{E}\enotes
\bar
\Notes\ibl{0}{b}{0}\uppz{b}\qb{0}{b}\uppz{b}\qb{0}{b}\uppz{b}\qb{0}{b}\tbl{0}\uppz{b}\qb{0}{b}\enotes
\Notes\ibl{0}{b}{0}\uppz{b}\qb{0}{b}\tbl{0}\uppz{b}\qb{0}{b}\enotes
\notes\ibl{0}{a}{0}\ibbl{0}{a}{0}\isluru{0}{c}\qb{0}{c}\qb{0}{b}\qb{0}{=a}\tbbl{0}\tbl{0}\tslur{0}{b}\qb{0}{b}\enotes
\bar
\notes\cl{e}\ds\enotes
\NOtes\zchar{-5}{\ppfftwenty sf}\isluru{0}{e}\hl{e}\enotes
\notes\tslur{0}{d}\cl{d}\ds\enotes
\bar
\Notes\qp\enotes
\NOtes\zchar{-5}{\ppfftwenty sf}\isluru{0}{c}\hl{c}\enotes
\notes\tslur{0}{b}\cl{b}\ds\enotes
\bar
\Notes\qp\enotes
\NOtes\zchar{-5}{\ppfftwenty sf}\isluru{0}{a}\hl{a}\enotes
\notes\tslur{0}{'G}\cl{G}\ds\enotes
\bar
\Notes\ibl{0}{'F}{0}\qb{0}{F}\qb{0}{F}\qb{0}{F}\tbl{0}\qb{0}{F}\enotes
\NOtes\loffset{0.5}{\ibl{0}{c}{9}}\roffset{0.5}{\tbl{0}}\hl{b}\enotes
\bar
\Notes\ibl{0}{'E}{1}\uppz{E}\qb{0}{E}\uppz{E}\qb{0}{E}\uppz{!c}\qb{0}{c}\tbl{0}\uppz{c}\qb{0}{c}\enotes
\Notes\ibl{0}{a}{0}\uppz{a}\qb{0}{=a}\uppz{a}\qb{0}{a}\uppz{b}\qb{0}{b}\tbl{0}\uppz{b}\qb{0}{b}\enotes
\bar
\Notes\ibl{0}{'E}{1}\uppz{E}\qb{0}{E}\uppz{E}\qb{0}{E}\uppz{!a}\qb{0}{_a}\tbl{0}\uppz{a}\qb{0}{a}\enotes
\Notes\ibl{0}{'F}{0}\uppz{F}\qb{0}{^F}\uppz{F}\qb{0}{F}\uppz{G}\qb{0}{G}\tbl{0}\uppz{G}\qb{0}{G}\enotes
\bar
\Notes\ibl{0}{'C}{1}\uppz{C}\qb{0}{C}\uppz{C}\qb{0}{C}\uppz{F}\qb{0}{=F}\tbl{0}\uppz{F}\qb{0}{F}\enotes
\Notes\ibl{0}{'D}{1}\uppz{D}\qb{0}{D}\uppz{D}\qb{0}{D}\uppz{E}\qb{0}{E}\tbl{0}\uppz{E}\qb{0}{E}\enotes
\bar
\NOtes\zchar{-5}{\ppfftwenty sf}\hl{=a}\zchar{-5}{\ppfftwenty sf}\hl{b}\enotes
\bar
\Notes\ibl{0}{c}{0}\uppz{c}\qb{0}{c}\uppz{c}\qb{0}{c}\uppz{c}\qb{0}{c}\tbl{0}\uppz{c}\qb{0}{c}\enotes
\NOtes\loffset{0.67}{\uppz{c}}\loffset{0.25}{\uppz{c}}\roffset{0.25}{\uppz{c}}\roffset{0.67}{\uppz{c}}\loffset{0.5}{\ibl{0}{d}{9}}\roffset{0.5}{\tbl{0}}\hl{c}\enotes
\bar
\Notes\lppz{'C}\qu{C}\uppz{D}\ql{D}\uppz{E}\ql{E}\uppz{E}\ql{=E}\enotes
\bar
\Notes\ql{'F}\qp\enotes
\NOtes\hpause\enotes
\bar
\Notes\ibl{0}{'F}{0}\qb{0}{F}\qb{0}{F}\qb{0}{F}\tbl{0}\qb{0}{F}\enotes
\NOtes\loffset{0.5}{\ibl{0}{'G}{9}}\roffset{0.5}{\tbl{0}}\hl{F}\enotes
\bar
\Notes\ql{'F}\qp\enotes
\NOtes\hpause\enotes
\bar
\Notes\ibl{0}{'F}{0}\qb{0}{F}\qb{0}{F}\qb{0}{F}\tbl{0}\qb{0}{F}\enotes
\NOtes\loffset{0.5}{\ibl{0}{'G}{9}}\roffset{0.5}{\tbl{0}}\hl{F}\enotes
\bar
\Notes\ql{'F}\qp\enotes
\NOtes\hpause\enotes
\setdoublebar
\endpiece
\end{music}

\subsubsection*{ロッシーニ: 弦楽のためのソナタ第3番 第3楽章より}
\begin{music}
\nostartrule
\setclef1{\bass}
\generalsignature{0}    
\generalmeter{\meterfrac24}
\parindent 0pt
\startbarno=41
\def\writebarno{\tenrm\the\barno\barnoadd}
\def\raisebarno{2\internote}
\def\shiftbarno{0.1\Interligne}
\systemnumbers
\startpiece\bigaccid
\notes\zchar{-8}{\mf}\ibl{0}{'C}{2}\ibbl{0}{C}{2}\upz{C}\qb{0}{C}\upz{D}\qb{0}{D}\upz{E}\qb{0}{E}\tbbl{0}\tbl{0}\upz{F}\qb{0}{F}\enotes
\notes\ibl{0}{'G}{2}\ibbl{0}{G}{2}\upz{G}\qb{0}{G}\upz{!a}\qb{0}{a}\upz{b}\qb{0}{b}\tbbl{0}\tbl{0}\upz{c}\qb{0}{c}\enotes
\bar
\notes\isluru{0}{c}\ibl{0}{b}{0}\ibbl{0}{b}{0}\qb{0}{b}\tslur{0}{d}\qb{0}{c}\upz{d}\qb{0}{d}\tbbl{0}\tbl{0}\upz{b}\qb{0}{b}\enotes
\notes\isluru{0}{a}\ibl{0}{'G}{-2}\ibbl{0}{G}{-2}\qb{0}{G}\tslur{0}{!b}\qb{0}{a}\upz{'G}\qb{0}{G}\tbbl{0}\tbl{0}\upz{F}\qb{0}{F}\enotes
\bar
\notes\ibl{0}{'E}{3}\ibbl{0}{E}{3}\upz{E}\qb{0}{E}\upz{G}\qb{0}{G}\upz{!c}\qb{0}{c}\tbbl{0}\tbl{0}\upz{b}\qb{0}{b}\enotes
\notes\ibl{0}{a}{-2}\ibbl{0}{a}{-2}\upz{a}\qb{0}{a}\upz{'G}\qb{0}{G}\upz{F}\qb{0}{F}\tbbl{0}\tbl{0}\upz{E}\qb{0}{E}\enotes
\bar
\notes\ibl{0}{'D}{3}\ibbl{0}{D}{3}\upz{D}\qb{0}{D}\upz{G}\qb{0}{G}\upz{b}\qb{0}{!b}\tbbl{0}\tbl{0}\upz{a}\qb{0}{a}\enotes
\notes\ibl{0}{'G}{-2}\ibbl{0}{G}{-2}\upz{G}\qb{0}{G}\upz{F}\qb{0}{F}\upz{E}\qb{0}{E}\tbbl{0}\tbl{0}\upz{D}\qb{0}{D}\enotes
\bar
\notes\ibl{0}{'C}{2}\ibbl{0}{C}{2}\upz{C}\qb{0}{C}\upz{D}\qb{0}{D}\upz{E}\qb{0}{E}\tbbl{0}\tbl{0}\upz{F}\qb{0}{F}\enotes
\notes\ibl{0}{'G}{2}\ibbl{0}{G}{2}\upz{G}\qb{0}{G}\upz{!a}\qb{0}{a}\upz{b}\qb{0}{b}\tbbl{0}\tbl{0}\upz{c}\qb{0}{c}\enotes
\bar
\notes\isluru{0}{c}\ibl{0}{b}{0}\ibbl{0}{b}{0}\qb{0}{b}\tslur{0}{d}\qb{0}{c}\upz{d}\qb{0}{d}\tbbl{0}\tbl{0}\upz{b}\qb{0}{b}\enotes
\notes\isluru{0}{a}\ibl{0}{'G}{-2}\ibbl{0}{G}{-2}\qb{0}{G}\tslur{0}{!b}\qb{0}{a}\upz{'G}\qb{0}{G}\tbbl{0}\tbl{0}\upz{F}\qb{0}{F}\enotes
\bar
\notes\ibl{0}{'E}{0}\ibbl{0}{E}{0}\upz{E}\qb{0}{E}\upz{!c}\qb{0}{c}\upz{'G}\qb{0}{G}\tbbl{0}\tbl{0}\upz{E}\qb{0}{E}\enotes
\notes\ibl{0}{'D}{0}\ibbl{0}{D}{0}\upz{D}\qb{0}{D}\upz{G}\qb{0}{G}\upz{F}\qb{0}{F}\tbbl{0}\tbl{0}\upz{D}\qb{0}{D}\enotes
\bar
\notes\ibl{0}{'D}{2}\ibbl{0}{D}{2}\upz{C}\qb{0}{C}\upz{!c}\qb{0}{c}\upz{'G}\qb{0}{G}\tbbl{0}\tbl{0}\upz{E}\qb{0}{E}\enotes
\notes\cu{'C}\ds\enotes
\bar
\notes\ibl{0}{'G}{2}\ibbl{0}{G}{2}\upz{G}\qb{0}{G}\upz{!a}\qb{0}{a}\upz{b}\qb{0}{b}\tbbl{0}\tbl{0}\upz{c}\qb{0}{c}\enotes
\notes\ibl{0}{b}{-2}\ibbl{0}{b}{-2}\upz{d}\qb{0}{d}\upz{b}\qb{0}{b}\upz{'G}\qb{0}{G}\tbbl{0}\tbl{0}\upz{!b}\qb{0}{b}\enotes
\bar
\notes\ibl{0}{c}{1}\ibbl{0}{c}{1}\upz{c}\qb{0}{c}\upz{d}\qb{0}{d}\upz{e}\qb{0}{e}\tbbl{0}\tbl{0}\upz{d}\qb{0}{d}\enotes
\notes\ibl{0}{c}{-2}\ibbl{0}{c}{-2}\upz{c}\qb{0}{c}\upz{b}\qb{0}{b}\upz{a}\qb{0}{a}\tbbl{0}\tbl{0}\upz{'G}\qb{0}{G}\enotes
\bar
\notes\ibl{0}{'F}{2}\ibbl{0}{F}{2}\upz{F}\qb{0}{^F}\upz{G}\qb{0}{G}\upz{!a}\qb{0}{a}\tbbl{0}\tbl{0}\upz{'G}\qb{0}{G}\enotes
\notes\ibl{0}{'D}{0}\ibbl{0}{D}{0}\upz{F}\qb{0}{F}\upz{D}\qb{0}{D}\upz{E}\qb{0}{E}\tbbl{0}\tbl{0}\upz{F}\qb{0}{F}\enotes
\bar
\notes\ibl{0}{'G}{2}\ibbl{0}{G}{2}\upz{G}\qb{0}{G}\upz{!a}\qb{0}{a}\upz{b}\qb{0}{b}\tbbl{0}\tbl{0}\upz{a}\qb{0}{a}\enotes
\notes\ibl{0}{'G}{-2}\ibbl{0}{G}{-2}\upz{G}\qb{0}{G}\upz{F}\qb{0}{=F}\upz{E}\qb{0}{E}\tbbl{0}\tbl{0}\upz{D}\qb{0}{D}\enotes
\bar
\notes\ibl{0}{'C}{2}\ibbl{0}{C}{2}\upz{C}\qb{0}{C}\upz{D}\qb{0}{D}\upz{E}\qb{0}{E}\tbbl{0}\tbl{0}\upz{F}\qb{0}{F}\enotes
\notes\ibl{0}{'G}{2}\ibbl{0}{G}{2}\upz{G}\qb{0}{G}\upz{!a}\qb{0}{a}\upz{b}\qb{0}{b}\tbbl{0}\tbl{0}\upz{c}\qb{0}{c}\enotes
\bar
\notes\isluru{0}{c}\ibl{0}{b}{0}\ibbl{0}{b}{0}\qb{0}{b}\tslur{0}{d}\qb{0}{c}\upz{d}\qb{0}{d}\tbbl{0}\tbl{0}\upz{b}\qb{0}{b}\enotes
\notes\isluru{0}{a}\ibl{0}{'G}{-2}\ibbl{0}{G}{-2}\qb{0}{G}\tslur{0}{!b}\qb{0}{a}\upz{'G}\qb{0}{G}\tbbl{0}\tbl{0}\upz{F}\qb{0}{F}\enotes
\bar
\notes\ibl{0}{'E}{0}\ibbl{0}{E}{0}\upz{E}\qb{0}{E}\upz{!c}\qb{0}{c}\upz{'G}\qb{0}{G}\tbbl{0}\tbl{0}\upz{E}\qb{0}{E}\enotes
\notes\ibl{0}{'D}{0}\ibbl{0}{D}{0}\upz{D}\qb{0}{D}\upz{G}\qb{0}{G}\upz{F}\qb{0}{F}\tbbl{0}\tbl{0}\upz{D}\qb{0}{D}\enotes
\bar
\notes\ibl{0}{'D}{2}\ibbl{0}{D}{2}\upz{C}\qb{0}{C}\upz{!c}\qb{0}{c}\upz{'G}\qb{0}{G}\tbbl{0}\tbl{0}\upz{E}\qb{0}{E}\enotes
\notes\cu{'C}\ds\enotes
\mulooseness=0
\setdoublebar\endpiece
\end{music}


\subsection{はね弓(2) リコシェ (仏: ricochet)}
\subsubsection*{シベリウス: ヴァイオリン協奏曲 ニ短調 第3楽章冒頭}
\begin{music}
\nostartrule
\setclef1{\bass}
\generalsignature{2}    
\generalmeter{\meterfrac34}
\parindent 0pt
\startbarno=1
\def\writebarno{\tenrm\the\barno\barnoadd}
\def\raisebarno{2\internote}
\def\shiftbarno{0.1\Interligne}
\systemnumbers
\startpiece\bigaccid
\notes\zchar{16}{\bf Allegro (\metron{\qu}{108-116})}\enotes
\Notes\ibl{0}{'D}{0}\qb{0}{D}\ibbl{0}{'D}{0}\enotes
\notes\qb{0}{'D}\tbl{0}\tbbl{0}\qb{0}{D}\enotes
\Notes\ibl{0}{'D}{0}\qb{0}{D}\ibbl{0}{'D}{0}\enotes
\notes\qb{0}{'D}\tbl{0}\tbbl{0}\qb{0}{D}\enotes
\Notes\ibl{0}{'D}{0}\qb{0}{D}\ibbl{0}{'D}{0}\enotes
\notes\qb{0}{'D}\tbl{0}\tbbl{0}\qb{0}{D}\enotes
\bar
\Notes\ibl{0}{'D}{0}\qb{0}{D}\ibbl{0}{'D}{0}\enotes
\notes\qb{0}{'D}\tbl{0}\tbbl{0}\qb{0}{D}\enotes
\Notes\ibl{0}{'D}{0}\qb{0}{D}\ibbl{0}{'D}{0}\enotes
\notes\qb{0}{'D}\tbl{0}\tbbl{0}\qb{0}{D}\enotes
\Notes\ibl{0}{'D}{0}\qb{0}{D}\ibbl{0}{'D}{0}\enotes
\notes\qb{0}{'D}\tbl{0}\tbbl{0}\qb{0}{D}\enotes
\bar
\Notes\ibl{0}{'D}{0}\qb{0}{D}\ibbl{0}{'D}{0}\enotes
\notes\qb{0}{'D}\tbl{0}\tbbl{0}\qb{0}{D}\enotes
\Notes\ibl{0}{'D}{0}\qb{0}{D}\ibbl{0}{'D}{0}\enotes
\notes\qb{0}{'D}\tbl{0}\tbbl{0}\qb{0}{D}\enotes
\Notes\ibl{0}{'D}{0}\qb{0}{D}\ibbl{0}{'D}{0}\enotes
\notes\qb{0}{'D}\tbl{0}\tbbl{0}\qb{0}{D}\enotes
\bar
\Notes\ibl{0}{'D}{0}\qb{0}{D}\ibbl{0}{'D}{0}\enotes
\notes\qb{0}{'D}\tbl{0}\tbbl{0}\qb{0}{D}\enotes
\Notes\ibl{0}{'D}{0}\qb{0}{D}\ibbl{0}{'D}{0}\enotes
\notes\qb{0}{'D}\tbl{0}\tbbl{0}\qb{0}{D}\enotes
\Notes\ibl{0}{'D}{0}\qb{0}{D}\ibbl{0}{'D}{0}\enotes
\notes\qb{0}{'D}\tbl{0}\tbbl{0}\qb{0}{D}\enotes
\setdoublebar
\endpiece
\end{music}

\subsubsection*{ロッシーニ: 歌劇「ウィリアム・テル」序曲より}
\begin{music}
\nostartrule
\setclef1{\bass}
\generalsignature{1}    
\generalmeter{\meterfrac24}
\parindent 0pt
\startbarno=243
\def\writebarno{\tenrm\the\barno\barnoadd}
\def\raisebarno{2\internote}
\def\shiftbarno{0.1\Interligne}
\systemnumbers
\startpiece\bigaccid
\notes\zchar{20}{\bf Allegro vivace (\metron{\qu}{152})}\enotes
\notes\qp\ds\zchar{10}{arco}\zchar{-6}{\pp}\ibl{0}{'E}{0}\ibbl{0}{E}{0}\zchar{13}{\downbow}\upz{E}\qb{0}{E}\tbbl{0}\tbl{0}\zchar{13}{\downbow}\upz{E}\qb{0}{E}\zchar{14}{\LARGE \bf H}\enotes
\bar
\Notes\ibl{0}{'E}{0}\zchar{9}{\upbow}\upz{E}\qb{0}{E}\enotes
\notes\ibbl{0}{'E}{0}\zchar{9}{\downbow}\upz{'E}\qb{0}{E}\tbbl{0}\tbl{0}\zchar{9}{\downbow}\upz{E}\qb{0}{E}\enotes
\Notes\ibl{0}{'E}{0}\upz{E}\qb{0}{E}\enotes
\notes\ibbl{0}{'E}{0}\upz{'E}\qb{0}{E}\tbbl{0}\tbl{0}\upz{E}\qb{0}{E}\enotes
\bar
\Notes\ibl{0}{'E}{0}\zchar{9}{\upbow}\upz{E}\qb{0}{E}\zchar{9}{\downbow}\upz{E}\qb{0}{E}\tbl{0}\zchar{9}{\upbow}\upz{E}\qb{0}{E}\enotes
\notes\ibl{0}{'E}{0}\ibbl{0}{E}{0}\zchar{9}{\downbow}\upz{E}\qb{0}{E}\tbbl{0}\tbl{0}\zchar{9}{\downbow}\upz{E}\qb{0}{E}\enotes
\bar
\Notes\ibl{0}{'E}{0}\upz{E}\qb{0}{E}\enotes
\notes\ibbl{0}{'E}{0}\upz{'E}\qb{0}{E}\tbbl{0}\tbl{0}\upz{E}\qb{0}{E}\enotes
\Notes\ibl{0}{'E}{0}\upz{E}\qb{0}{E}\enotes
\notes\ibbl{0}{'E}{0}\upz{'E}\qb{0}{E}\tbbl{0}\tbl{0}\upz{E}\qb{0}{E}\enotes
\bar
\Notes\ibu{0}{'B}{0}\lpz{B}\qb{0}{B}\lpz{B}\qb{0}{B}\tbu{0}\lpz{B}\qb{0}{B}\enotes
\notes\ibl{0}{'E}{0}\ibbl{0}{E}{0}\upz{E}\qb{0}{E}\tbbl{0}\tbl{0}\upz{E}\qb{0}{E}\enotes
\bar
\Notes\ibl{0}{'E}{0}\upz{E}\qb{0}{E}\enotes
\notes\ibbl{0}{'E}{0}\upz{'E}\qb{0}{E}\tbbl{0}\tbl{0}\upz{E}\qb{0}{E}\enotes
\Notes\ibl{0}{'E}{0}\upz{E}\qb{0}{E}\enotes
\notes\ibbl{0}{'E}{0}\upz{'E}\qb{0}{E}\tbbl{0}\tbl{0}\upz{E}\qb{0}{E}\enotes
\bar
\Notes\ibl{0}{'E}{0}\upz{E}\qb{0}{E}\upz{E}\qb{0}{E}\tbl{0}\upz{E}\qb{0}{E}\enotes
\notes\zchar{-6}{\ff}\ibl{0}{'E}{0}\ibbl{0}{E}{0}\upz{E}\qb{0}{E}\tbbl{0}\tbl{0}\upz{E}\qb{0}{E}\enotes
\bar
\Notes\ibu{0}{'B}{0}\lpz{B}\qb{0}{B}\enotes
\notes\ibbu{0}{'B}{0}\lpz{'B}\qb{0}{B}\tbbu{0}\tbu{0}\lpz{B}\qb{0}{B}\enotes
\Notes\ibu{0}{'B}{0}\lpz{B}\qb{0}{B}\enotes
\notes\ibbu{0}{'B}{0}\lpz{'B}\qb{0}{B}\tbbu{0}\tbu{0}\lpz{B}\qb{0}{B}\enotes
\bar
\Notes\ibl{0}{'E}{0}\upz{E}\qb{0}{E}\upz{E}\qb{0}{E}\tbl{0}\upz{E}\qb{0}{E}\enotes
\notes\zchar{-6}{\pp}\ibl{0}{'E}{0}\ibbl{0}{E}{0}\upz{E}\qb{0}{E}\tbbl{0}\tbl{0}\upz{E}\qb{0}{E}\enotes
\bar
\Notes\ibl{0}{'E}{0}\upz{E}\qb{0}{E}\enotes
\notes\ibbl{0}{'E}{0}\upz{'E}\qb{0}{E}\tbbl{0}\tbl{0}\upz{E}\qb{0}{E}\enotes
\Notes\ibl{0}{'E}{0}\upz{E}\qb{0}{E}\enotes
\notes\ibbl{0}{'E}{0}\upz{'E}\qb{0}{E}\tbbl{0}\tbl{0}\upz{E}\qb{0}{E}\enotes
\bar
\Notes\ibl{0}{'E}{0}\upz{E}\qb{0}{E}\upz{E}\qb{0}{E}\tbl{0}\upz{E}\qb{0}{E}\enotes
\notes\ibl{0}{'E}{0}\ibbl{0}{E}{0}\upz{E}\qb{0}{E}\tbbl{0}\tbl{0}\upz{E}\qb{0}{E}\enotes
\bar
\Notes\ibl{0}{'E}{0}\upz{E}\qb{0}{E}\enotes
\notes\ibbl{0}{'E}{0}\upz{'E}\qb{0}{E}\tbbl{0}\tbl{0}\upz{E}\qb{0}{E}\enotes
\Notes\ibl{0}{'E}{0}\upz{E}\qb{0}{E}\enotes
\notes\ibbl{0}{'E}{0}\upz{'E}\qb{0}{E}\tbbl{0}\tbl{0}\upz{E}\qb{0}{E}\enotes
\bar
\Notes\ibu{0}{'B}{0}\lpz{B}\qb{0}{B}\lpz{B}\qb{0}{B}\tbu{0}\lpz{B}\qb{0}{B}\enotes
\notes\ibl{0}{'E}{0}\ibbl{0}{E}{0}\upz{E}\qb{0}{E}\tbbl{0}\tbl{0}\upz{E}\qb{0}{E}\enotes
\bar
\Notes\ibl{0}{'E}{0}\upz{E}\qb{0}{E}\enotes
\notes\ibbl{0}{'E}{0}\upz{'E}\qb{0}{E}\tbbl{0}\tbl{0}\upz{E}\qb{0}{E}\enotes
\Notes\ibl{0}{'E}{0}\upz{E}\qb{0}{E}\enotes
\notes\ibbl{0}{'E}{0}\upz{'E}\qb{0}{E}\tbbl{0}\tbl{0}\upz{E}\qb{0}{E}\enotes
\bar
\Notes\ibl{0}{'E}{0}\upz{E}\qb{0}{E}\upz{E}\qb{0}{E}\tbl{0}\upz{E}\qb{0}{E}\enotes
\notes\zchar{-6}{\ff}\ibl{0}{'E}{0}\ibbl{0}{E}{0}\upz{E}\qb{0}{E}\tbbl{0}\tbl{0}\upz{E}\qb{0}{E}\enotes
\bar
\Notes\ibu{0}{'B}{0}\lpz{B}\qb{0}{B}\enotes
\notes\ibbu{0}{'B}{0}\lpz{'B}\qb{0}{B}\tbbu{0}\tbu{0}\lpz{B}\qb{0}{B}\enotes
\Notes\ibu{0}{'B}{0}\lpz{B}\qb{0}{B}\enotes
\notes\ibbu{0}{'B}{0}\lpz{'B}\qb{0}{B}\tbbu{0}\tbu{0}\lpz{B}\qb{0}{B}\enotes
\bar
\Notes\ibl{0}{'E}{0}\upz{E}\qb{0}{E}\upz{E}\qb{0}{E}\tbl{0}\upz{E}\qb{0}{E}\ds\enotes
\setdoublebar
\endpiece
\end{music}

%\subsubsection*{ロッシーニ: 歌劇「ウィリアム・テル」序曲より}


\subsection{はね弓(3) 一弓スタッカート (one bow staccato)}
%\subsubsection*{マーラー: 交響曲第4番 第1楽章より}
\subsubsection*{マーラー: 交響曲第4番 ト長調 第1楽章より}
\begin{music}
\nostartrule
\setclef1{\bass}
\generalsignature{1}    
\generalmeter{\meterfrac44}
\parindent 0pt
\startbarno=21
\def\writebarno{\tenrm\the\barno\barnoadd}
\def\raisebarno{2\internote}
\def\shiftbarno{0.1\Interligne}
\systemnumbers
\startpiece\bigaccid
\notes\zchar{18}{\bf Bed\"{a}chtig, nicht eilen}\enotes
\notes\zchar{-8}{\pp \it \ \ legg.}\qs\zchar{10}{\upbow}\isluru{0}{'F}\ibl{0}{D}{2}\ibbl{0}{D}{2}\upz{D}\qb{0}{D}\upz{E}\qb{0}{E}\tbbl{0}\tbl{0}\upz{F}\qb{0}{F}\enotes
\notes\ibl{0}{'G}{2}\ibbl{0}{G}{2}\upz{G}\qb{0}{G}\upz{!a}\qb{0}{a}\upz{b}\qb{0}{b}\tslur{0}{e}\tbbl{0}\tbl{0}\upz{c}\qb{0}{c}\enotes
\bar
\NOtes\zchar{11}{\downbow}\isluru{0}{a}\usf{'G}\ql{^G}\tslur{0}{!b}\qlp{a}\enotes
\notes\loffset{1.5}{\cbreath}\isluru{0}{e}\ibl{0}{c}{2}\ibbl{0}{c}{2}\upz{c}\qb{0}{^c}\tbbl{0}\tbl{0}\upz{d}\qb{0}{d}\enotes
\notes\ibl{0}{d}{0}\ibbl{0}{d}{0}\upz{e}\qb{0}{e}\upz{d}\qb{0}{d}\upz{c}\qb{0}{c}\tslur{0}{f}\tbbl{0}\tbl{0}\upz{d}\qb{0}{d}\enotes
\bar
\NOtes\zchar{15}{\downbow}\isluru{0}{f}\usf{c}\ql{^c}\tslur{0}{c}\qlp{b}\enotes
\notes\loffset{1.5}{\cbreath}\isluru{0}{c}\ibl{0}{a}{-2}\ibbl{0}{a}{-2}\upz{a}\qb{0}{a}\tbbl{0}\tbl{0}\upz{'G}\qb{0}{G}\enotes
\notes\ibl{0}{'F}{2}\ibbl{0}{F}{2}\upz{F}\qb{0}{F}\upz{G}\qb{0}{G}\upz{!a}\qb{0}{a}\tslur{0}{d}\tbbl{0}\tbl{0}\upz{b}\qb{0}{b}\enotes
\bar
\notes\ibu{0}{'C}{1}\ibbu{0}{C}{1}\lpz{C}\qb{0}{C}\lpz{B}\qb{0}{B}\lpz{C}\qb{0}{C}\tbbu{0}\tbu{0}\lpz{D}\qb{0}{D}\enotes
\notes\ibl{0}{'E}{1}\ibbl{0}{E}{1}\upz{E}\qb{0}{E}\upz{D}\qb{0}{^D}\upz{E}\qb{0}{E}\tbbl{0}\tbl{0}\upz{F}\qb{0}{F}\enotes
\notes\ibl{0}{'G}{0}\ibbl{0}{G}{0}\upz{G}\qb{0}{G}\upz{F}\qb{0}{F}\upz{!a}\qb{0}{a}\tbbl{0}\tbl{0}\upz{'G}\qb{0}{G}\enotes
\notes\ibl{0}{'F}{-1}\ibbl{0}{F}{-1}\upz{F}\qb{0}{F}\upz{!a}\qb{0}{a}\upz{'G}\qb{0}{G}\tbbl{0}\tbl{0}\upz{E}\qb{0}{E}\enotes
\bar
\Notes\loffset{1.0}{\zchar{-4}{\pp}}\islurd{0}{'C}\qu{^C}\tslur{0}{D}\qup{D}\enotes
\setdoublebar
\endpiece
\end{music}



\subsection{半音階}
レプレ、メンデルスゾーン3,4

\subsection{仕上げ: 芸術系+高速系の名曲}

\subsubsection*{ベートーヴェン: 交響曲第9番ニ短調「合唱付き」 第4楽章よりレチタチーヴォ+歓喜主題}

