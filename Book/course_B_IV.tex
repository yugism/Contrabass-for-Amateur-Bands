\section{コースB: 第IVポジション}
\begin{center}
\begin{tabular}{|lcl|}
\hline
この章の基礎練習 & : & 1. 開放弦の練習 2. 「\ref{half_scale}」「\ref{1st_scale}」「\ref{2nd_scale}」の音階練習 3. 「弦楽セレナーデ」\\
この章の修了課題 & : & 1. 「\ref{4th_scale}」の音階練習を正しい音程で暗譜して演奏できる\\ 
               &   & 2. 「マイスタージンガー」を正しい音程で暗譜して演奏できる\\\hline
\end{tabular}
\end{center}

\subsection{第IVポジションの位置}
\begin{flushleft}
\begin{minipage}{280pt}
\ \ \ 親指をネックの一番下まで降ろしてください。その親指の正面を2の指
で押さえます。これが4thポジションです。1と4は2に合わせます。指の間隔は
ハーフポジションと比べてだいぶ狭くなります。このポジションのD、A、E線
で取れる音は、細い方の隣の弦の1stポジションであることを覚えてください。
\subsection{第IVポジションで取れる音}
\begin{music}
\nostartrule
\parindent 0pt
\setclef1{\bass}  
\startpiece
\notes\enotes
\Notes\zchar{18}{G線}\zchar{13}{\bf 1}\wh{d}\zchar{14}{\bf 2}\wh{^d}\zchar{14}{\bf 4}\wh{e}\enotes
\doublebar
\Notes\zchar{13}{\bf 1}\wh{d}\zchar{14}{\bf 2}\wh{_e}\zchar{14}{\bf 4}\wh{=e}\enotes
\doublebar
\Notes\zchar{18}{D線}\zchar{11}{\bf 1}\wh{a}\zchar{11}{\bf 2}\wh{^a}\zchar{11}{\bf 4}\wh{b}\enotes
\doublebar
\Notes\zchar{11}{\bf 1}\wh{a}\zchar{11}{\bf 2}\wh{_b}\zchar{11}{\bf 4}\wh{=b}\enotes
\setdoublebar
\endpiece
\startpiece
\notes\enotes
\Notes\zchar{14}{A線}\zchar{9}{\bf 1}\wh{'E}\zchar{9}{\bf 2}\wh{F}\zchar{9}{\bf 4}\wh{^F}\enotes
\doublebar
\Notes\zchar{9}{\bf 1}\wh{'E}\zchar{9}{\bf 2}\wh{F}\zchar{9}{\bf 4}\wh{_G}\enotes
\doublebar
\Notes\zchar{14}{E線}\zchar{9}{\bf 1}\wh{'B}\zchar{9}{\bf 2}\wh{C}\zchar{9}{\bf 4}\wh{^C}\enotes
\doublebar
\Notes\zchar{9}{\bf 1}\wh{'B}\zchar{9}{\bf 2}\wh{C}\zchar{9}{\bf 4}\wh{_D}\enotes
\setdoublebar
\endpiece
\end{music}
\end{minipage}
\hfill
\begin{minipage}{65pt}
\addtocounter{figure}{1}
\begin{center}
\includegraphics[height=6.5cm]{../Vol1/Pics/Position/4th_2.epsi}\\
{\flushleft\small 図\thefigure : 4th\(=\)ネックの一番下\\}
\end{center}
\end{minipage}
\hfill
\begin{minipage}{80pt}
\addtocounter{figure}{1}
\begin{center}
\includegraphics[height=6.5cm]{../Vol1/Pics/Position/4th_3.epsi}\\
{\flushleft\small 図\thefigure : 既出ポジションとの位置関係\\}
\end{center}
\end{minipage}
\end{flushleft}
\subsection{音階練習 \label{4th_scale}}
\begin{music}
\nostartrule
\parindent 0pt
\setclef1{\bass}  
\generalmeter{\meterC}
\generalsignature{-3}    
\startpiece
\notes\zchar{18}{変ホ長調(Es-dur)音階}\enotes
\NOtes\zchar{15}{half}\ovbkt{f}{3}{2}\zchar{7}{\bf 1}\ql{'E}\zchar{8}{\bf 4}\ql{F}\zchar{9}{\bf 0}\ql{G}\zchar{10}{\bf 1}\ql{!a}\enotes
\bar
\NOtes\zchar{17}{II}\ovbkt{'b}{1.1}{0}\zchar{11}{\bf 1}\ql{!b}\zchar{12}{\bf 4}\ql{c}\zchar{19}{IV}\ovbkt{'d}{3.4}{0}\zchar{13}{\bf 1}\ql{!d}\zchar{14}{\bf 2}\ql{e}\enotes
\bar
\NOtes\zchar{14}{\bf 2}\ql{e}\zchar{13}{\bf 1}\ql{d}\zchar{17}{II}\ovbkt{'b}{1.1}{0}\zchar{12}{\bf 4}\ql{!c}\zchar{11}{\bf 1}\ql{b}\enotes
\bar
\NOtes\zchar{16}{half}\ovbkt{'a}{3}{-3}\zchar{10}{\bf 1}\ql{!a}\zchar{9}{\bf 0}\ql{'G}\zchar{8}{\bf 4}\ql{F}\zchar{7}{\bf 1}\ql{E}\enotes
\setdoublebar\endpiece
\end{music}
\subsection{ハーフ、第I、第II、第IVポジションで弾ける名曲}
\input{Pieces/meister2}
