\section*{まえがき}
本書はアマチュア楽団での初心者育成に役立つことを第一の目的にしています。\\
\indent 今日、各地の市民楽団にコントラバス演奏を楽しむアマチュア奏者の
方々がいらっしゃいます。私が接した限りでは、その大半は学校の吹奏楽部、
オーケストラ、室内合奏団で先輩から習うことで弾き方を覚えたという方々で
占められているようです。こうした学生の楽団はアマチュアベーシストのゆりかごと言えましょう。\\
\indent かく言う私もこのゆりかごで育てられた一人です。私がコントラバス
を覚えた学生オケ\footnote{「オーケストラ」の略。}も他の多くの団体と同
様にベーシストの人数が不足がちなオケでしたので、新入生にも年度末の定期
演奏会のメインプログラム\footnote {その演奏会で最大規模の曲のこと。オー
ケストラの場合は交響曲であることが多い。} に乗る
\footnote{「舞台上での合奏に参加する」という意味の隠語。}こ
とが期待されていました。新人育成方針もこうした事情を反映して、まずは指
づかいを早期に習得させ、あとは合奏練習に参加させて実地で鍛える、という
ものでした。そして実際、左手の指づかいを一通りマスターしたところで合奏
練習に加わることが許され、演奏会本番ではメインプログラムの交響曲を含めて2
曲に乗ることができました。\\
\indent このとき先輩から指定されて勉強したのがフランツ・シマンドル
(Frantz Simandl)の教則本です。アマチュア楽団において、初心者育成に使わ
れている教則本としてはこのシマンドルやルートヴィヒ・シュトライヒャー
(Ludwig Streicher)によるものに定評があります。これらのテキストは左手の
指づかいの習得について整った教育体系を提供しており、プロフェッショナル
の演奏家を目指す学生の要求にも応え得るだけの豊富な分量と詳細な記述とを
誇る名著です。\\
\indent しかし、さしもの名著もアマチュア、特に1年目でメインプログラム
に乗ることが求められているような中・小規模の楽団の初心者新人には分量が
多すぎるきらいがあります。また、含まれている練習曲の多くは音楽的な魅力に乏し
く、プロを目指すつもりのない初心者にはかなりの忍耐を強いるものと言わざ
るを得ません。忍耐しかねた貴重な新人がやめてしまったら、ベーシストの少
ないアマチュア楽団にとっては打撃です。このように、専門家養成に定評ある
名著もアマチュアのニーズには必ずしも合致しているとは言えない、というの
が私の実感です。\\
\indent こうした点を踏まえて、本書はこれらの教則本の特長である左手指づ
かいの教育体系を継承しつつも、アマチュア楽団の実情に合った内容・分
量となるよう以下の3原則に基づいて執筆されています。

\begin{center}
\begin{tabular}{ll}
\multicolumn{2}{l}{\bf 1. コンパクト}\\
 ----- & 左手指づかいの習得に特化することで分量を最小限に抑える。\\
       & その代わりに少ない曲を暗譜するまでさらう。\\
       & 分量が少ないので消化不良感がない。初年度のうちにメインプログラムに乗ることを想定。\\
\multicolumn{2}{l}{\bf 2. 楽しく}\\
 ----- & 習熟段階ごとに大作曲家の名曲を教材にして楽しく深く学ぶ。\\
       & 特定の楽派に偏ることなく、様々な作曲家の曲を採用する。\\
       & 単調な練習曲に飽きた初心者が楽団をやめたくなるのを防止する。\\
\multicolumn{2}{l}{\bf 3. 練習曲のリサイクル}\\
 ----- & 弾き捨ての練習曲では報われない。\\
       & 練習曲を集合させると「一生ものの基礎練習(第\ref{scales}章)」になるよう構成。\\
\end{tabular}
\end{center}

結果として、「こんな教本で育てたい」「こんな教本で教わりたかった」と私
自身が魅力を感じられるだけのものが出来上がったと思います。本書によって、
一生ものの趣味としてコントラバスを楽しむ仲間が増えることを心より願って
やみません。\\

\begin{small}
\begin{flushright}
\begin{minipage}{100pt}
\begin{flushleft}
平成15年8月\\
多摩の自宅にて\\
\end{flushleft}
\center{柚木克之}
\end{minipage}
\end{flushright}
\end{small}

%やって見せ  言って聞かせてやらせてみ  ほめてやらねば人は動かぬ(山本五十六)
%g\"{o}tterfunken の飛び散る瞬間を一つでも多く曲中に(小林研一郎)
