\section{予備知識}
\subsection{音名}

通常、日本のオーケストラや吹奏楽団ではプロアマを問わずドイツ式の音名を使って意思疎通を図ります。この慣習にスムーズに適応できるよう、本書ではドイツ式の音名を用います。\\

\subsubsection*{音名の対応表}

\begin{center}
\begin{small}
\begin{tabular}{l|llllllll}
         & \multicolumn{8}{c}{音名}\\
\hline
日本     & ハ & ニ & ホ & ヘ & ト & イ & ロ & ハ\\
イタリア & Do (ド) & Re (レ) & Mi (ミ) & Fa (ファ) & Sol (ソ) & La (ラ) & Si (シ) & Do (ド)\\
英語圏   & C & D & E & F & G & A & B & C \\ 
ドイツ   & C (ツェー)  & D (デー)  & E (エー)   & F (エフ)   & G (ゲー) & A (アー) & H (ハー) & C (ツェー) \\ 
\end{tabular}
\end{small}
\end{center}

\subsubsection*{音名と読み方}
ほぼアルファベット順ですが、「シ」の音のみ「H (ハー)」になります。

\begin{music}
\nostartrule
\parindent 0pt
\setclef1{\bass}  
\startextract
\NOTEs\zchar{-7}{\ \ \ C}\zchar{-10}{\mbox{ツェー}}\wh{'C}\enotes
\doublebar
\NOTEs\zchar{-7}{\ \ \ D}\zchar{-10}{\mbox{デー}}\wh{'D}\enotes
\doublebar
\NOTEs\zchar{-7}{\ \ \ E}\zchar{-10}{\mbox{エー}}\wh{'E}\enotes
\doublebar
\NOTEs\zchar{-7}{\ \ \ F}\zchar{-10}{\mbox{エフ}}\wh{'F}\enotes
\doublebar
\NOTEs\zchar{-7}{\ \ \ G}\zchar{-10}{\mbox{ゲー}}\wh{'G}\enotes
\doublebar
\NOTEs\zchar{-7}{\ \ \ A}\zchar{-10}{\mbox{アー}}\wh{a}\enotes
\doublebar
\NOTEs\zchar{-7}{\ \ \ H}\zchar{-10}{\mbox{ハー}}\wh{b}\enotes
\doublebar
\NOTEs\zchar{-7}{\ \ \ C}\zchar{-10}{\mbox{ツェー}}\wh{c}\enotes
\setdoublebar
\endextract
\end{music}

\subsubsection*{\(\flat\) が付いた音}
もとの音に"es"を付けます。ただし、フラットの付いたHの音だけは「B (ベー)」と呼びます。

\begin{music}
\nostartrule
\parindent 0pt
\setclef1{\bass}  
\startextract
\NOTEs\zchar{-7}{\ \ Ces}\zchar{-10}{\mbox{ツェス}}\wh{'_C}\enotes
\doublebar
\NOTEs\zchar{-7}{\ \ Des}\zchar{-10}{\mbox{デス}}\wh{'_D}\enotes
\doublebar
\NOTEs\zchar{-7}{\ \ Es}\zchar{-10}{\mbox{エス}}\wh{'_E}\enotes
\doublebar
\NOTEs\zchar{-7}{\ \ Fes}\zchar{-10}{\mbox{フェス}}\wh{'_F}\enotes
\doublebar
\NOTEs\zchar{-7}{\ \ Ges}\zchar{-10}{\mbox{ゲス}}\wh{'_G}\enotes
\doublebar
\NOTEs\zchar{-7}{\ \ As}\zchar{-10}{\mbox{アス}}\wh{_a}\enotes
\doublebar
\NOTEs\zchar{-7}{\ \ B}\zchar{-10}{\mbox{ベー}}\wh{_b}\enotes
\doublebar
\NOTEs\zchar{-7}{\ \ Ces}\zchar{-10}{\mbox{ツェス}}\wh{_c}\enotes
\setdoublebar
\endextract
\end{music}

\subsubsection*{\(\sharp\) が付いた音}
もとの音に"is"を付けます。

\begin{music}
\nostartrule
\parindent 0pt
\setclef1{\bass}  
\startextract
\NOTEs\zchar{-7}{\ \ Cis}\zchar{-10}{\mbox{ツィス}}\wh{'^C}\enotes
\doublebar
\NOTEs\zchar{-7}{\ \ Dis}\zchar{-10}{\mbox{ディス}}\wh{'^D}\enotes
\doublebar
\NOTEs\zchar{-7}{\ \ Eis}\zchar{-10}{\mbox{エイス}}\wh{'^E}\enotes
\doublebar
\NOTEs\zchar{-7}{\ \ Fis}\zchar{-10}{\mbox{フィス}}\wh{'^F}\enotes
\doublebar
\NOTEs\zchar{-7}{\ \ Gis}\zchar{-10}{\mbox{ギス}}\wh{'^G}\enotes
\doublebar
\NOTEs\zchar{-7}{\ \ Ais}\zchar{-10}{\mbox{アイス}}\wh{^a}\enotes
\doublebar
\NOTEs\zchar{-7}{\ \ His}\zchar{-10}{\mbox{ヒス}}\wh{^b}\enotes
\doublebar
\NOTEs\zchar{-7}{\ \ Cis}\zchar{-10}{\mbox{ツィス}}\wh{^c}\enotes
\setdoublebar
\endextract
\end{music}

\subsection{各部の名称}

下図に楽器と弓の各部分の名称をまとめました。今の時点で全部覚える必要はありません。例えば「手の平で毛箱を包み込むように」という説明を読んで「どの部品のことだろう?」と思ったときに、この図を見てその都度確認すれば十分です。
