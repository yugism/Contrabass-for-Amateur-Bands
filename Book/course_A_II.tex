\section{第IIポジション}
\begin{center}
\begin{tabular}{|lcl|}
\hline
この章の基礎練習 & : & 1. 開放弦の練習 2. 「\ref{half_scale}」「\ref{1st_scale}」の音階練習 3. マーラー\\
この章の修了課題 & : & 1. 「\ref{2nd_scale}」の音階練習を正しい音程で暗譜して演奏できる\\ 
               &   & 2. 「弦楽セレナーデ」を正しい音程で暗譜して演奏できる\\
\hline
\end{tabular}
\end{center}
\begin{flushleft}
\begin{minipage}{320pt}
\subsection{第IIポジションの位置}
\ \ \ 第IIIポジションでは1で押さえていた音を4の指で取ります。第III、第IVポジションと比べると指の間隔が少
し広くなります。
\subsection{第IIポジションで取れる音}

\begin{music}
\nostartrule
\parindent 0pt
\setclef1{\bass}  
\startpiece
\notes\enotes
\Notes\zchar{16}{G線}\zchar{10}{\bf 1}\wh{^a}\zchar{11}{\bf 2}\wh{b}\zchar{12}{\bf 4}\wh{c}\enotes
\doublebar
\Notes\zchar{11}{\bf 1}\wh{_b}\zchar{11}{\bf 2}\wh{=b}\zchar{12}{\bf 4}\wh{c}\enotes
\doublebar
\Notes\zchar{16}{D線}\zchar{9}{\bf 1}\wh{'F}\zchar{9}{\bf 2}\wh{^F}\zchar{9}{\bf 4}\wh{G}\enotes
\doublebar
\Notes\zchar{9}{\bf 1}\wh{'F}\zchar{9}{\bf 2}\wh{_G}\zchar{9}{\bf 4}\wh{=G}\enotes
\setdoublebar
\endpiece
\startpiece
\notes\enotes
\Notes\zchar{14}{A線}\zchar{9}{\bf 1}\wh{'C}\zchar{9}{\bf 2}\wh{^C}\zchar{9}{\bf 4}\wh{D}\enotes
\doublebar
\Notes\zchar{9}{\bf 1}\wh{'C}\zchar{9}{\bf 2}\wh{_D}\zchar{9}{\bf 4}\wh{=D}\enotes
\doublebar
\Notes\zchar{14}{E線}\zchar{9}{\bf 1}\wh{G}\zchar{9}{\bf 2}\wh{^G}\zchar{9}{\bf 4}\wh{'A}\enotes
\doublebar
\Notes\zchar{9}{\bf 1}\wh{G}\zchar{9}{\bf 2}\wh{'_A}\zchar{9}{\bf 4}\wh{=A}\enotes
\setdoublebar
\endpiece
\end{music}
\end{minipage}
\hfill
\begin{minipage}{80pt}
\addtocounter{figure}{1}
\begin{center}
\includegraphics[width=3cm]{../Vol1/Pics/Position/2nd.epsi}\\
{\flushleft\small 図\thefigure : 第IIポジションと既出ポジションの位置関係\\}
\end{center}
\end{minipage}
\end{flushleft}


\subsection{音階練習 \label{2nd_scale}}
\begin{music}
\nostartrule
\parindent 0pt
\setclef1{\bass}
\generalmeter{\meterC}  
\startpiece
\notes\zchar{18}{ハ長調(C-dur)音階}\enotes
\NOtes\ovbkt{f}{3.5}{0}\zchar{9}{\bf 2}\zchar{14}{IV}\qu{'C}\zchar{9}{\bf 0}\ql{D}\zchar{9}{\bf 1}\ql{E}\zchar{9}{\bf 2}\ql{F}\enotes
\bar
\NOtes\ovbkt{f}{1.1}{4}\zchar{9}{\bf 1}\zchar{15}{III}\ql{'G}\zchar{10}{\bf 4}\ql{!a}\zchar{17}{II}\ovbkt{'a}{3.5}{0}\zchar{11}{\bf 2}\ql{!b}\zchar{12}{\bf 4}\ql{c}\enotes
\bar
\NOtes\zchar{12}{\bf 4}\ql{c}\zchar{11}{\bf 2}\ql{b}\ovbkt{g}{1.1}{-3}\zchar{10}{\bf 4}\zchar{15}{III}\ql{a}\zchar{9}{\bf 1}\ql{'G}\enotes
\bar
\NOtes\ovbkt{f}{3.5}{0}\zchar{9}{\bf 2}\zchar{14}{IV}\ql{'F}\zchar{9}{\bf 1}\ql{E}\zchar{9}{\bf 0}\ql{D}\zchar{9}{\bf 2}\qu{C}\enotes
\setdoublebar\endpiece
\end{music}

\subsection{第IIポジションまでで弾ける名曲}
\documentclass{jarticle}
\usepackage{musixdoc}
\startmuflex\makeindex

\newcommand{\flag}[2]{\zchar{#2}{$\circ$}\zchar{#1}{\tiny +}}
\newcommand{\press}[2]{\zchar{#2}{$\circ$}\zchar{#1}{\small \|}}

\begin{document}

\subsubsection*{�١��ȡ�������: �������9�֥�ûĴ�ֹ羧�դ��� ��4�ھϤ��}

\begin{music}
\setclef1{\bass}
\generalsignature{2}    
\generalmeter{\meterC}
\nostartrule
\parindent 0pt
\startbarno=92
\def\writebarno{\tenrm\the\barno\barnoadd}
\def\raisebarno{2\internote}
\def\shiftbarno{0.1\Interligne}
\systemnumbers
\startpiece\bigaccid
\qspace
\notes\zchar{14}{\bf Allegro assai (\metron{\hu}{80})}\zchar{-5}{\p}\enotes
\NOtes\isluru{0}{'F}\hl{F}\enotes
\Notes\ql{'G}\tslur{0}{!a}\ql{a}\enotes
\bar
\Notes\isluru{0}{a}\ql{a}\ql{'G}\ql{F}\tslur{0}{E}\ql{E}\enotes
\bar
\NOtes\isluru{0}{'D}\hl{D}\enotes
\Notes\ql{'E}\tslur{0}{F}\ql{F}\enotes
\bar
\NOtes\isluru{0}{'F}\qlp{F}\enotes
\notes\tslur{0}{'E}\cl{E}\enotes
\Notes\hl{'E}\enotes
\bar
\NOtes\isluru{0}{'F}\hl{F}\enotes
\Notes\ql{'G}\tslur{0}{!a}\ql{a}\enotes
\bar
\Notes\isluru{0}{a}\ql{a}\ql{'G}\ql{F}\tslur{0}{E}\ql{E}\enotes
\bar
\NOtes\isluru{0}{'D}\hl{D}\enotes
\Notes\ql{'E}\tslur{0}{F}\ql{F}\enotes
\bar
\NOtes\isluru{0}{'E}\qlp{E}\enotes
\notes\tslur{0}{'D}\cl{D}\enotes
\Notes\hl{'D}\enotes
\bar
\NOtes\isluru{0}{'E}\hl{E}\enotes
\Notes\ql{'F}\tslur{0}{D}\ql{D}\enotes
\bar
\Notes\isluru{0}{'E}\ql{E}\enotes
\notes\ibl{0}{'F}{0}\qb{0}{F}\tbl{0}\qb{0}{G}\enotes
\Notes\ql{'F}\tslur{0}{D}\ql{D}\enotes
\bar
\Notes\isluru{0}{'F}\ql{E}\enotes
\notes\ibl{0}{'F}{0}\qb{0}{F}\tbl{0}\qb{0}{G}\enotes
\Notes\ql{'F}\tslur{0}{F}\ql{E}\enotes
\bar
\Notes\zchar{-5}{\it cresc.}\isluru{0}{'D}\ql{D}\ql{E}\tslur{0}{E}\qu{A}\isluru{0}{F}\ql{F}\enotes
\bar
\Notes\zchar{-5}{\p}\tslur{0}{'F}\ql{F}\isluru{0}{F}\ql{F}\ql{G}\tslur{0}{!a}\ql{a}\enotes
\bar
\Notes\isluru{0}{a}\ql{a}\ql{'G}\ql{F}\tslur{0}{E}\ql{E}\enotes
\bar
\NOtes\isluru{0}{'D}\hl{D}\enotes
\Notes\ql{'E}\tslur{0}{F}\ql{F}\enotes
\bar
\NOtes\isluru{0}{'E}\qlp{E}\enotes
\notes\tslur{0}{'D}\cl{D}\enotes
\Notes\hl{'D}\enotes
\bar
\NOtes\isluru{0}{'E}\hl{E}\enotes
\Notes\ql{'F}\tslur{0}{D}\ql{D}\enotes
\bar
\Notes\isluru{0}{'E}\ql{E}\enotes
\notes\ibl{0}{'F}{0}\qb{0}{F}\tbl{0}\qb{0}{G}\enotes
\Notes\ql{'F}\tslur{0}{D}\ql{D}\enotes
\bar
\Notes\isluru{0}{'F}\ql{E}\enotes
\notes\ibl{0}{'F}{0}\qb{0}{F}\tbl{0}\qb{0}{G}\enotes
\Notes\ql{'F}\tslur{0}{F}\ql{E}\enotes
\bar
\Notes\zchar{-5}{\it cresc.}\isluru{0}{'D}\ql{D}\ql{E}\tslur{0}{E}\qu{A}\isluru{0}{F}\ql{F}\enotes
\bar
\Notes\zchar{-5}{\p}\tslur{0}{'F}\ql{F}\isluru{0}{F}\ql{F}\ql{G}\tslur{0}{!a}\ql{a}\enotes
\bar
\Notes\isluru{0}{a}\ql{a}\ql{'G}\ql{F}\tslur{0}{E}\ql{E}\enotes
\bar
\NOtes\isluru{0}{'D}\hl{D}\enotes
\Notes\ql{'E}\tslur{0}{F}\ql{F}\enotes
\bar
\NOtes\isluru{0}{'E}\qlp{E}\enotes
\notes\tslur{0}{'D}\cl{D}\enotes
\Notes\ql{'D}\zchar{3}{\LARGE (}\sk\ql{!a}\zchar{3}{\LARGE )}\sk\enotes
\setdoublebar
\endpiece
\end{music}

\endmuflex
\end{document}

\subsubsection*{ヨハン・シュトラウスII : 「皇帝円舞曲」より}
\begin{music}
\nostartrule
\setclef1{\bass}
\generalsignature{0}    
\generalmeter{\meterfrac34}
\parindent 0pt
\startbarno=52
\def\writebarno{\tenrm\the\barno\barnoadd}
\def\raisebarno{2\internote}
\def\shiftbarno{0.1\Interligne}
\systemnumbers
\startpiece\bigaccid
\notes\zchar{22}{\bf (Tempo di Valse)}\zchar{-4}{\ff \ {\it marcato}}\enotes
\Notes\zchar{17}{\downbow}\zchar{15}{\bf 4}\usfz{c}\zchar{-9}{II}\hlp{c}\enotes
\bar
\Notes\zchar{17}{\upbow}\zchar{14}{\bf 2}\usfz{b}\hlp{b}\enotes
\bar
\notes\zchar{14}{\upbow}\zchar{11}{\bf 2}\ql{b}\zchar{14}{\downbow}\zchar{12}{\bf 4}\ql{c}\zchar{12}{\upbow}\zchar{9}{\bf 1}\zchar{-4}{III}\ql{'G}\enotes
\bar
\Notes\zchar{12}{\downbow}\zchar{10}{\bf 4}\hlp{a}\enotes
\bar
\Notes\zchar{15}{\upbow}\zchar{12}{\bf 4}\zchar{-4}{II}\hlp{c}\enotes
\bar
\Notes\zchar{13}{\downbow}\zchar{11}{\bf 2}\hlp{b}\enotes
\bar
\notes\zchar{14}{\upbow}\zchar{11}{\bf 2}\ql{b}\zchar{14}{\downbow}\zchar{12}{\bf 4}\ql{c}\zchar{13}{\upbow}\zchar{10}{\bf 1}\zchar{-4}{IV}\ql{a}\enotes
\bar
\Notes\zchar{18}{\downbow}\zchar{15}{\bf 4}\isluru{0}{e}\hlp{e}\enotes
\bar
\notes\tslur{0}{e}\hl{e}\zchar{13}{\upbow}\zchar{10}{\bf 1}\ql{a}\enotes
\alaligne
\Notes\zchar{15}{\downbow}\zchar{13}{\bf 1}\hlp{d}\enotes
\bar
\notes\zchar{15}{\upbow}\zchar{12}{\bf 4}\zchar{-4}{II}\ql{c}\zchar{13}{\downbow}\zchar{11}{\bf 2}\ql{b}\zchar{13}{\upbow}\zchar{10}{\bf 4}\zchar{-4}{III}\ql{a}\enotes
\bar
\notes\zchar{12}{\downbow}\zchar{10}{\bf 4}\hl{a}\zchar{12}{\downbow}\zchar{9}{\bf 1}\ql{'G}\enotes
\bar
\notes\zchar{12}{\upbow}\zchar{9}{\bf 4}\ql{'E}\zchar{12}{\downbow}\zchar{9}{\bf 1}\ql{G}\zchar{15}{\upbow}\zchar{12}{\bf 1}\ql{!c}\enotes
\bar
\Notes\zchar{14}{\downbow}\zchar{12}{\bf 4}\usfz{'G}\zchar{-4}{II}\hlp{G}\enotes
\bar
\Notes\zchar{15}{\upbow}\zchar{12}{\bf 1}\usfz{'G}\hup{!G}\enotes
\bar
\notes\zchar{16}{\upbow}\zchar{13}{\bf 1}\usfz{a}\qu{'C}\zchar{12}{\downbow}\zchar{10}{\bf 4}\zchar{-5}{III}\ql{E}\zchar{13}{\upbow}\zchar{10}{\bf 1}\ql{G}\enotes
\bar
\notes\zchar{14}{\downbow}\zchar{12}{\bf 1}\ql{c}\qp\qp\enotes
\setdoublebar
\endpiece
\end{music}

%\subsubsection*{モーツァルト: レクイエム ニ短調 K.626 より''I. Introitus''冒頭}
\begin{music}
\nostartrule
\setclef1{\bass}
\generalsignature{-1}    
\generalmeter{\meterC}
\parindent 0pt
\startbarno=1
\def\writebarno{\tenrm\the\barno\barnoadd}
\def\raisebarno{2\internote}
\def\shiftbarno{0.1\Interligne}
\systemnumbers
\startpiece\bigaccid
\notes\zchar{18}{\bf Adagio}\enotes
\notes\cl{'D}\ds\cl{F}\ds\cl{E}\ds\cl{G}\ds\enotes
\bar
\notes\cl{'F}\ds\cu{^C}\ds\cl{D}\ds\cl{D}\ds\enotes
\bar
\notes\cu{'=C}\ds\cu{!^G}\ds\cu{'A}\ds\cl{D}\ds\enotes
\bar
\notes\cu{=G}\ds\cu{'A}\ds\cl{F}\ds\cl{E}\ds\enotes
\bar
\notes\cl{'D}\ds\cl{G}\ds\cu{C}\ds\cu{A}\ds\enotes
\bar
\notes\cl{'D}\ds\cl{D}\ds\cl{E}\ds\cl{E}\ds\enotes
\bar
\notes\cl{a}\ds\cl{'=G}\ds\cl{F}\ds\cl{E}\ds\enotes
\bar
\notes\cl{'D}\zchar{2}{\Huge ( } \ds\cl{D}\ds\cl{D}\ds\cu{^C}\ds\zchar{2}{\Huge )}\enotes
\setdoublebar\endpiece
\end{music}

\subsubsection*{ヘンデル: オラトリオ「メサイア」より「ハレルヤ」}
\begin{music}
\nostartrule
\setclef1{\bass}
\generalsignature{2}    
\generalmeter{\meterC}
\parindent 0pt
\startbarno=8
\def\writebarno{\tenrm\the\barno\barnoadd}
\def\raisebarno{2\internote}
\def\shiftbarno{0.1\Interligne}
\systemnumbers
\startpiece\bigaccid
\notes\zchar{18}{\bf Allegro}\enotes
\NOtes\zchar{13}{\downbow}\zchar{11}{\bf 4}\zchar{-4}{III}\qlp{a}\enotes
\Notes\zchar{16}{\upbow}\zchar{13}{\bf 2}\cl{c}\ibl{0}{d}{-4}\zchar{16}{\downbow}\zchar{14}{\bf 4}\qb{0}{d}\tbl{0}\zchar{14}{\upbow}\zchar{11}{\bf 4}\qb{0}{a}\qp\enotes
\bar
\NOtes\zchar{13}{\downbow}\zchar{11}{\bf 4}\qlp{a}\enotes
\Notes\zchar{16}{\upbow}\zchar{13}{\bf 2}\cl{c}\ibl{0}{d}{-4}\zchar{16}{\downbow}\zchar{14}{\bf 4}\qb{0}{d}\tbl{0}\zchar{14}{\upbow}\zchar{11}{\bf 4}\qb{0}{a}\ds\ibl{0}{c}{0}\ibbl{0}{c}{0}\zchar{16}{\downbow}\zchar{13}{\bf 2}\qb{0}{c}\tbl{0}\tbbl{0}\zchar{16}{\upbow}\zchar{13}{\bf 2}\qb{0}{c}\enotes
\bar
\Notes\ibl{0}{d}{-4}\zchar{16}{\downbow}\zchar{14}{\bf 4}\qb{0}{d}\tbl{0}\zchar{14}{\upbow}\zchar{11}{\bf 4}\qb{0}{a}\ds\ibl{0}{c}{0}\ibbl{0}{c}{0}\zchar{16}{\downbow}\zchar{13}{\bf 2}\qb{0}{c}\tbl{0}\tbbl{0}\zchar{16}{\upbow}\zchar{13}{\bf 2}\qb{0}{c}\ibl{0}{d}{-4}\zchar{16}{\downbow}\zchar{14}{\bf 4}\qb{0}{d}\tbl{0}\zchar{14}{\upbow}\zchar{11}{\bf 4}\qb{0}{a}\ds\zchar{16}{\upbow}\zchar{13}{\bf 2}\cl{c}\enotes
\bar
\Notes\ibl{0}{d}{-2}\zchar{16}{\downbow}\zchar{14}{\bf 4}\qb{0}{d}\tbl{0}\zchar{16}{\upbow}\zchar{13}{\bf 2}\qb{0}{c}\zchar{14}{\downbow}\zchar{12}{\bf 4}\zchar{-4}{IV}\ql{b}\zchar{14}{\upbow}\zchar{11}{\bf 1}\ql{a}\qp\enotes
\bar
\NOtes\zchar{13}{\downbow}\zchar{11}{\bf 1}\zchar{-4}{\ff}\hl{a}\enotes
\Notes\zchar{15}{\upbow}\zchar{12}{\bf 4}\ql{b}\zchar{16}{\upbow}\zchar{13}{\bf 2}\zchar{-4}{III}\ql{c}\enotes
\bar
\Notes\ibl{0}{'F}{-2}\zchar{16}{\downbow}\zchar{14}{\bf 4}\qb{0}{!d}\tbl{0}\zchar{13}{\upbow}\zchar{10}{\bf 1}\qb{0}{'D}\enotes
\NOtes\zchar{16}{\downbow}\zchar{14}{\bf 4}\hl{d}\enotes
\Notes\zchar{16}{\upbow}\zchar{13}{\bf 2}\ql{c}\zchar{19}{\LARGE \bf A}\enotes
\bar
\NOtes\zchar{14}{\downbow}\zchar{12}{\bf 4}\zchar{-4}{IV}\hl{b}\enotes
\Notes\zchar{14}{\upbow}\zchar{11}{\bf 1}\ql{a}\ds\zchar{-4}{\f}\ibl{0}{c}{0}\ibbl{0}{c}{0}\zchar{16}{\downbow}\zchar{13}{\bf 2}\zchar{-9}{III}\qb{0}{c}\tbl{0}\tbbl{0}\zchar{16}{\upbow}\zchar{13}{\bf 2}\qb{0}{c}\enotes
\bar
\Notes\ibl{0}{d}{-4}\zchar{16}{\downbow}\zchar{14}{\bf 4}\qb{0}{d}\tbl{0}\zchar{14}{\upbow}\zchar{11}{\bf 4}\qb{0}{a}\ds\ibl{0}{c}{0}\ibbl{0}{c}{0}\zchar{16}{\downbow}\zchar{13}{\bf 2}\qb{0}{c}\tbl{0}\tbbl{0}\zchar{16}{\upbow}\zchar{13}{\bf 2}\qb{0}{c}\ibl{0}{d}{-4}\zchar{16}{\downbow}\zchar{14}{\bf 4}\qb{0}{d}\tbl{0}\zchar{14}{\upbow}\zchar{11}{\bf 4}\qb{0}{a}\ds\ibl{0}{c}{0}\ibbl{0}{c}{0}\zchar{16}{\downbow}\zchar{13}{\bf 2}\qb{0}{c}\tbl{0}\tbbl{0}\zchar{16}{\upbow}\zchar{13}{\bf 2}\qb{0}{c}\enotes
\bar
\Notes\ibl{0}{d}{-4}\zchar{16}{\downbow}\zchar{14}{\bf 4}\qb{0}{d}\tbl{0}\zchar{14}{\upbow}\zchar{11}{\bf 4}\qb{0}{a}\ds\ibl{0}{c}{0}\ibbl{0}{c}{0}\zchar{16}{\downbow}\zchar{13}{\bf 2}\qb{0}{c}\tbl{0}\tbbl{0}\zchar{16}{\upbow}\zchar{13}{\bf 2}\qb{0}{c}\ibl{0}{d}{-4}\zchar{16}{\downbow}\zchar{14}{\bf 4}\qb{0}{d}\tbl{0}\zchar{14}{\upbow}\zchar{11}{\bf 4}\qb{0}{a}\qp\enotes
\bar
\NOtes\zchar{11}{\downbow}\zchar{9}{\bf 1}\zchar{-5}{\ff}\hl{'D}\enotes
\Notes\zchar{12}{\upbow}\zchar{9}{\bf 4}\ql{'E}\zchar{12}{\upbow}\zchar{9}{\bf 2}\zchar{-4}{II}\ql{F}\enotes
\bar
\Notes\ibl{0}{'D}{-5}\zchar{11}{\downbow}\zchar{9}{\bf 4}\qb{0}{G}\tbl{0}\zchar{12}{\upbow}\zchar{9}{\bf 1}\qb{0}{!G}\enotes
\NOtes\zchar{11}{\downbow}\zchar{9}{\bf 4}\hl{'G}\enotes
\Notes\zchar{12}{\upbow}\zchar{9}{\bf 2}\ql{'F}\enotes
\bar
\NOtes\zchar{11}{\downbow}\zchar{9}{\bf 4}\zchar{-4}{III}\hl{'E}\enotes
\Notes\zchar{12}{\upbow}\zchar{9}{\bf 1}\ql{'D}\ds\zchar{-6}{\f}\ibl{0}{F}{0}\ibbl{0}{F}{0}\zchar{11}{\downbow}\zchar{9}{\bf 2}\zchar{-10}{II}\qb{0}{F}\tbl{0}\tbbl{0}\zchar{12}{\upbow}\zchar{9}{\bf 2}\qb{0}{F}\enotes
\bar
\Notes\ibl{0}{'G}{-4}\zchar{11}{\downbow}\zchar{9}{\bf 4}\qb{0}{G}\tbl{0}\zchar{12}{\upbow}\zchar{9}{\bf 4}\qb{0}{D}\ds\ibl{0}{F}{0}\ibbl{0}{F}{0}\zchar{11}{\downbow}\zchar{9}{\bf 2}\qb{0}{F}\tbl{0}\tbbl{0}\zchar{12}{\upbow}\zchar{9}{\bf 2}\qb{0}{F}\ibl{0}{G}{-4}\zchar{11}{\downbow}\zchar{9}{\bf 4}\qb{0}{G}\tbl{0}\zchar{12}{\upbow}\zchar{9}{\bf 4}\qb{0}{D}\ds\ibl{0}{F}{0}\ibbl{0}{F}{0}\zchar{11}{\downbow}\zchar{9}{\bf 2}\qb{0}{F}\tbl{0}\tbbl{0}\zchar{12}{\upbow}\zchar{9}{\bf 2}\qb{0}{F}\enotes
\bar
\Notes
\ibl{0}{'G}{-4}\zchar{11}{\downbow}\zchar{9}{\bf 4}\qb{0}{G}\tbl{0}\zchar{12}{\upbow}\zchar{9}{\bf 4}\qb{0}{D}\ds\ibl{0}{F}{0}\ibbl{0}{F}{0}\zchar{11}{\downbow}\zchar{9}{\bf 2}\qb{0}{F}\tbl{0}\tbbl{0}\zchar{12}{\upbow}\zchar{9}{\bf 2}\qb{0}{F}\ibl{0}{G}{-4}\zchar{11}{\downbow}\zchar{9}{\bf 4}\qb{0}{G}\tbl{0}\zchar{12}{\upbow}\zchar{9}{\bf 4}\qb{0}{D}\qp\enotes
\setdoublebar
\endpiece
\end{music}

