\section{コースB: 第IIポジション}
\begin{center}
\begin{tabular}{|lcl|}
\hline
この章の基礎練習 & : & 1. 開放弦の練習 2. 「\ref{half_scale}」「\ref{1st_scale}」の音階練習 3. マーラー\\
この章の修了課題 & : & 1. 「\ref{2nd_scale}」の音階練習を正しい音程で暗譜して演奏できる\\ 
               &   & 2. 「弦楽セレナーデ」を正しい音程で暗譜して演奏できる\\
\hline
\end{tabular}
\end{center}
\begin{flushleft}
\begin{minipage}{320pt}
\subsection{第IIポジションの位置}
\ \ \ 第IIポジションは第Iの半音上に位置します。ハーフポジションでは
4で押さえていた音を1の指で取ります。ハーフ、第Iと比べると指の間隔が少
し狭くなります。
\subsection{第IIポジションで取れる音}

\begin{music}
\nostartrule
\parindent 0pt
\setclef1{\bass}  
\startpiece
\notes\enotes
\Notes\zchar{16}{G線}\zchar{10}{\bf 1}\wh{^a}\zchar{11}{\bf 2}\wh{b}\zchar{12}{\bf 4}\wh{c}\enotes
\doublebar
\Notes\zchar{11}{\bf 1}\wh{_b}\zchar{11}{\bf 2}\wh{=b}\zchar{12}{\bf 4}\wh{c}\enotes
\doublebar
\Notes\zchar{16}{D線}\zchar{9}{\bf 1}\wh{'F}\zchar{9}{\bf 2}\wh{^F}\zchar{9}{\bf 4}\wh{G}\enotes
\doublebar
\Notes\zchar{9}{\bf 1}\wh{'F}\zchar{9}{\bf 2}\wh{_G}\zchar{9}{\bf 4}\wh{=G}\enotes
\setdoublebar
\endpiece
\startpiece
\notes\enotes
\Notes\zchar{14}{A線}\zchar{9}{\bf 1}\wh{'C}\zchar{9}{\bf 2}\wh{^C}\zchar{9}{\bf 4}\wh{D}\enotes
\doublebar
\Notes\zchar{9}{\bf 1}\wh{'C}\zchar{9}{\bf 2}\wh{_D}\zchar{9}{\bf 4}\wh{=D}\enotes
\doublebar
\Notes\zchar{14}{E線}\zchar{9}{\bf 1}\wh{G}\zchar{9}{\bf 2}\wh{^G}\zchar{9}{\bf 4}\wh{'A}\enotes
\doublebar
\Notes\zchar{9}{\bf 1}\wh{G}\zchar{9}{\bf 2}\wh{'_A}\zchar{9}{\bf 4}\wh{=A}\enotes
\setdoublebar
\endpiece
\end{music}
\end{minipage}
\hfill
\begin{minipage}{80pt}
\addtocounter{figure}{1}
\begin{center}
\includegraphics[width=3cm]{../Vol1/Pics/Position/2nd.epsi}\\
{\flushleft\small 図\thefigure : 第IIポジションと既出ポジションの位置関係\\}
\end{center}
\end{minipage}
\end{flushleft}


\subsection{音階練習 \label{2nd_scale}}
\begin{music}
\nostartrule
\parindent 0pt
\setclef1{\bass}
\generalmeter{\meterC}  
\startpiece
\notes\zchar{14}{ハ長調(C-dur)音階}\enotes
\NOtes\zchar{9}{\bf 2}\qu{'C}\zchar{9}{\bf 0}\ql{D}\zchar{9}{\bf 1}\ql{E}\zchar{9}{\bf 2}\ql{F}\enotes
\bar
\NOtes\zchar{9}{\bf 0}\ql{'G}\zchar{10}{\bf 1}\ql{!a}\zchar{17}{II}\ovbkt{'a}{3.5}{0}\zchar{11}{\bf 2}\ql{!b}\zchar{12}{\bf 4}\ql{c}\enotes
\bar
\NOtes\zchar{12}{\bf 4}\ql{c}\zchar{11}{\bf 2}\ql{b}\zchar{10}{\bf 1}\ql{a}\zchar{9}{\bf 0}\ql{'G}\enotes
\bar
\NOtes\zchar{9}{\bf 2}\ql{'F}\zchar{9}{\bf 1}\ql{E}\zchar{9}{\bf 0}\ql{D}\zchar{9}{\bf 2}\qu{C}\enotes
\setdoublebar\endpiece
\end{music}

\subsection{第IIポジションまでで弾ける名曲}
\subsubsection*{チャイコフスキー: 弦楽セレナーデ ハ長調 第1楽章 「ソナチナ形式の小品」より}

\begin{music}
\nostartrule
\setclef1{\bass}
\generalsignature{0}    
\generalmeter{\meterfrac68}
\parindent 0pt
\def\writebarno{\tenrm\the\barno\barnoadd}
\def\raisebarno{2\internote}
\def\shiftbarno{0.1\Interligne}
\systemnumbers
\startpiece\bigaccid
\notes\zchar{22}{\bf Andante non troppo}\enotes
\NOtes\zchar{-7}{\f}\zchar{-3}{I}\zchar{14}{\downbow}\zchar{12}{\bf 1}\usf{a}\qlp{a}\zchar{11}{\bf 4}\zchar{-6}{II}\unbkt{D}{1.2}{0}\usf{'G}\qlp{G}\enotes
\bar
\NOtes\zchar{-6}{\it sempre marcatissimo}\zchar{13}{\downbow}\usf{'G}\qlp{F}\enotes
\notes\ibl{0}{'E}{0}\zchar{11}{I}\ovbkt{!c}{1.8}{0}\zchar{18}{\upbow}\zchar{15}{\bf 1}\qb{0}{'EF}\tbl{0}\qb{0}{E}\enotes
\bar
\NOtes\zchar{12}{II}\ovbkt{!d}{1.15}{5}\zchar{6}{\bf 4}\qlp{'D}\zchar{9}{\bf 4}\qlp{G}\enotes
\bar
\NOtes\zchar{-6}{half}\unbkt{D}{1.4}{0}\zchar{9}{\bf 1}\qlp{'^G}\enotes
\notes\ibl{0}{!a}{-3}\qb{0}{a=!'G}\tbl{0}\zchar{-6}{I}\unbkt{!D}{7.5}{0}\zchar{9}{\bf 2}\qb{0}{'F}\enotes
\bar
\NOtes\zchar{-6}{\icresc}\zchar{9}{\downbow}\qlp{'E}\enotes
\Notes\xtuplet{2}{'E}\ibl{0}{D}{1}\zchar{9}{\downbow}\zchar{6}{\bf 0}\qb{0}{D}\tbl{0}\zchar{10}{\upbow}\zchar{7}{\bf 1}\qb{0}{!'E}\enotes
\bar
\NOtes\isluru{0}{'F}\zchar{9}{\downbow}\qlp{F}\enotes
\Notes\isluru{1}{'G}\zchar{21}{\upbow}\zchar{18}{\bf 4}\zchar{14}{II}\ovbkt{!f}{1.5}{-7}\ql{'G}\enotes
\notes\ibbl{0}{'G}{-6}\tslur{0}{G}\tslur{1}{G}\qb{0}{G}\zchar{-6}{\tcresc}\tbbl{0}\tbl{0}\zchar{-5}{\ff}\zchar{15}{\upbow}\zchar{12}{\bf 1}\qb{0}{C}\enotes
\alaligne 
\NOtes\zchar{-5}{\ppfftwenty sf}\zchar{10}{\downbow}\qup{'C}\zchar{-5}{\ppfftwenty sf}\islurd{0}{C}\zchar{10}{\downbow}\qup{C}\enotes
\generalmeter{\meterfrac24}\changecontext 
\notes\ibbl{0}{'C}{3}\tslur{0}{C}\qb{0}{C}\unbkt{!`G}{4.8}{6}\zchar{14}{\upbow}\zchar{11}{\bf 2}\usf{!'G}\qb{0}{C}\zchar{-9}{I}\zchar{14}{\downbow}\zchar{11}{\bf 0}\usf{G}\qb{0}{D}\zchar{14}{\upbow}\zchar{11}{\bf 1}\usf{G}\qb{0}{E}\zchar{11}{\bf 2}\usf{G}\qb{0}{F}\zchar{11}{\bf 0}\usf{G}\qb{0}{G}\zchar{12}{\bf 1}\usf{'A}\qb{0}{A}\tbbl{0}\tbl{0}\zchar{-6}{II}\unbkt{!D}{4.5}{0}\zchar{13}{\bf 2}\usf{!b}\qb{0}{!b}\enotes
\generalmeter{\meterfrac68}\changecontext
\NOtes\zchar{-7}{\ff \ \it marcatissimo}\zchar{14}{\downbow}\zchar{12}{\bf 4}\qlp{cb}\enotes
\bar
\NOtes\zchar{-6}{I}\unbkt{!D}{1.5}{0}\zchar{12}{\downbow}\zchar{10}{\bf 1}\qlp{a}\enotes
\notes\ibl{0}{'G}{2}\zchar{12}{\upbow}\zchar{9}{\bf 0}\qb{0}{G!a}\tbl{0}\zchar{-6}{II}\unbkt{!D}{0.2}{0}\zchar{12}{\bf 4}\qb{0}{c}\enotes
\bar
\NOtes\isluru{0}{'G}\zchar{12}{\downbow}\zchar{9}{\bf 4}\qlp{G}\enotes
\Notes\tslur{0}{'G}\ql{G}\zchar{10}{\upbow}\cl{F}\enotes
\bar
\NOtes\zchar{-7}{I}\unbkt{!C}{2}{0}\zchar{9}{\bf 1}\qlp{'E}\enotes
\notes\ibl0{'E}{1}\qb0{EE}\tbl0\qb0{!a}\enotes
\bar
\NOtes\zchar{-6}{II}\unbkt{!D}{0.15}{0}\zchar{11}{\downbow}\zchar{9}{\bf 4}\qlp{'G}\enotes
\Notes\xtuplet{2}{'E}\ibl{0}{F}{-2}\zchar{-7}{I}\unbkt{!D}{0.85}{-6}\zchar{10}{\downbow}\zchar{8}{\bf 2}\qb{0}{'F}\tbl{0}\zchar{9}{\upbow}\qb0{E}\enotes
\bar
\NOtes\zchar{-6}{II}\zchar{9}{\downbow}\zchar{7}{\bf 4}\qlp{'D}\enotes
\Notes\isluru{0}{'G}\zchar{11}{\upbow}\ql{G}\enotes
\notes\ibbl{0}{'G}{-6}\tslur{0}{G}\qb{0}{G}\tbbl{0}\tbl{0}\zchar{9}{\upbow}\qb{0}{C}\enotes
\bar
\NOtes\zchar{10}{\downbow}\qup{'C}\enotes
\notes\zchar{-5}{\fff}\zchar{12}{\downbow}\cl{!c}\ds\ds\enotes
\mulooseness=0
\setdoublebar\endpiece
\end{music}


